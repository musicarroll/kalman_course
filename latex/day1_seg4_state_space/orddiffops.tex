% orddiffops.tex
\begin{frame}
Ordinary Differential Operators
 \begin{itemize}
  \item Linear operators on a metric space of differentiable functions.
  \item The derivative of a function $f$ at the point $t\in [a,b]$ is the limit as $s\rightarrow
 t$ of the difference quotient,
 \[\frac{f(s)-f(t)}{s-t},\]
assuming this limit exists.
\item Note that division by $s-t$ is scalar multiplication by
$ \frac{1}{s-t}$.  This implies that we are dealing with a vector
space over the reals.
\end{itemize}
\end{frame}
% orddiffeqs.tex
\begin{frame}
Ordinary Differential Operators
 \begin{itemize}
\item We shall be concerned here with the space of infinitely differentiable, real-valued functions defined on the closed real interval $[a,b]$:
 $C^\infty[a,b]$.
 \item This will give rise to deterministic differential
 equations.
 \item Later we will define derivatives of stochastic processes in
 order to deal with stochastic differential equations.  You cannot
 take the derivative (or even integrate) a random process in the
 ordinary sense.
\end{itemize}
\end{frame}
% orddiffeqs.tex
\begin{frame}
Ordinary Differential Operators
 \begin{itemize}
 \item Since the derivatives of all orders are assumed to exist
 for all functions $f\in C^\infty[a,b]$, the limit of the
 difference quotient always exists for such functions.
 \item Thus, we assign to each $f$ a new function $Df$ given by
\begin{equation}\label{deriv}
  [Df](t) = \lim_{s\rightarrow t}\frac{f(s)-f(t)}{s-t}
\end{equation}
 \item This assignment defines a transformation
 \[D:C^\infty[a,b]\rightarrow C^\infty[a,b]\]
  that is a linear operator.
\end{itemize}
\end{frame}
\begin{frame}
Ordinary Differential Operators
 \begin{itemize}
    \item $D$ is linear: $D[af+bg]=a[Df] + b[Dg]$
    \item Sometimes we get lax and simply write $Df(t)$ instead of $[Df](t)$.  However, keep in mind that $Df(t)$ says "Differntiate $f$ and then evaluate the resulting function at the point $t$."
    \item $D$ obeys all the usual rules of differntial calculus such as the product rule and the chain rule:
\[D(fg)=[Df]g+f[Dg]\]
\[D(f\circ g) = [Df]\circ g \cdot Dg\]
 \end{itemize}
\end{frame}
\begin{frame}
Ordinary Differential Operators: Examples
 \begin{itemize}
  \item If $\id$ is the identity function, i.e., $\id(t)=t$, then
    $D\id=1$. (You probably know this as $\frac{dt}{dt}=1$; Note that 1 is the function whose value at every $t$ is 1.)
  \item Using the product rule, we have
  \begin{align*}
    D\id^2 &= D[\id \cdot \id]\\
           &= D\id \cdot \id + \id \cdot D\id\\
           &= 1 \cdot \id + \id \cdot 1\\
           &= \id + \id\\
           &= 2\id
  \end{align*}
(In more familiar notation this is just $\frac{dt^2}{dt}=2t$.)
 \end{itemize}
\end{frame}
% orddiffeqs.tex
\begin{frame}
Ordinary Differential Operators
 \begin{itemize}
 \item Note that we sometimes write $d/dt$ or $\frac{d}{dt}$ for
 the operator $D$ when we want to emphasize the independent
 variable $t$.  Or, instead of $\frac{dy}{dt}$ we write simply $\dot{y}$.
 \item However, that is really unnecessary, adds no new information, and is just extra
 baggage.
 \item Nevertheless, we shall do it sometimes because it may be more familiar.
\end{itemize}
\end{frame}
% orddiffeqs.tex
\begin{frame}
Ordinary Differential Operators
 \begin{itemize}
 \item Since there is really only one generating operator $D$ that we are interested in here,
 we are less interested in multiplication than we are in function
 composition or iteration of this operator. For instance,
 $D^2f=D(Df)$.
 \item By induction we have $D^n=\underbrace{D\circ\cdots \circ D}_n$.
 \item Just as real-valued functions inherit algebraic operations from those of their range
 spaces, so too do differential operators:
 \[\left[aD^m+bD^n\right](f)=a(D^mf)+b(D^nf)\]
\end{itemize}
\end{frame}
% orddiffeqs.tex
\begin{frame}
Ordinary Differential Operators
 \begin{itemize}
 \item We can therefore generate polynomials in $D$:
 \[p(D)=a_nD^n+\cdots +a_1D+a_oI,\]
 where $I$ is the identity operator on $C^\infty[a,b]$.  We
 usually just drop the identity operator, since it amounts to multiplying by 1.
 \item Such polynomials in $D$ are still linear operators.
 \item Thus, we have rediscovered the group
 ring $\Real(\Diff)$, where $\Diff$ in this case is the infinite cyclic
 group generated by the differential operator $D$.  We'll call
 this group ring 
 the \textit{\textbf{polynomial ring of differential operators}}.
\end{itemize}
\end{frame}
