% coupled_damposcs.tex

\begin{frame}
	
	\onslide<1->{
		\frametitle{Coupled ODEs} 
		\framesubtitle{Coupled Damped Harmonic Oscillator}  
	}
	\begin{itemize}
		\onslide<2->{
			\item Two damped harmonic oscillators can be coupled via a third spring between the masses.
		}
		\onslide<3->{
			\item Consider two masses, $m_1$ and $m_2$, attached to springs with spring constants $k_1$ and $k_2$, respectively. 
		}
		\onslide<4->{
			\begin{itemize}
				\item The two masses are connected by a coupling spring with a spring constant $k_3$. The damping coefficients for the two masses are denoted as $b_1$ and $b_2$, respectively.
			\end{itemize}
		}
		\onslide<5->{
			\item A schematic drawing of this system is depicted on the next slide 
		}
	\end{itemize}
	\onslide<6->{\center --------------------------------------------------------------------}
	
\end{frame}



\begin{frame}
	
	\onslide<1->{
		\frametitle{Coupled ODEs} 
		\framesubtitle{Coupled Damped Harmonic Oscillator}  
	}
	\begin{center}
		\includegraphics[width=.9\textwidth,height=.6\textheight]{\dayonesegfour/coupled_damped_harmosc.png}
	\end{center}
	
	\onslide<2->{\center --------------------------------------------------------------------}
	
\end{frame}

\begin{frame}
	
	\onslide<1->{
		\frametitle{Coupled ODEs} 
		\framesubtitle{Coupled Damped Harmonic Oscillator}  
	}
	\begin{itemize}
		\onslide<2->{
			\item We can define the state variables as follows:
			\begin{itemize}
				\item $x_1$ = displacement of mass $m_1$ from its equilibrium position
				\item $x_2$ = displacement of mass $m_2$ from its equilibrium position
				\item $v_1$ = velocity of mass $m_1$
				\item $v_2$ = velocity of mass $m_2$
			\end{itemize}
			
		}
		\onslide<3->{
			\item The state space equations can be written as:
			
			\[
			\frac{{dx_1}}{{dt}} = v_1
			\]
			\[
			\frac{{dx_2}}{{dt}} = v_2
			\]
			\[
			\frac{{dv_1}}{{dt}} = \frac{{-k_1 \cdot x_1 - k_3 \cdot (x_1 - x_2) - b_1 \cdot v_1}}{{m_1}}
			\]
			\[
			\frac{{dv_2}}{{dt}} = \frac{{-k_2 \cdot x_2 + k_3 \cdot (x_1 - x_2) - b_2 \cdot v_2}}{{m_2}}
			\]
		}
	\end{itemize}
	\onslide<4->{\center --------------------------------------------------------------------}
	
\end{frame}


\begin{frame}
	
	\onslide<1->{
		\frametitle{Coupled ODEs} 
		\framesubtitle{Coupled Damped Harmonic Oscillator}  
	}
	\begin{itemize}
		\onslide<2->{
			\item In matrix form, the state space model can be written as:
			
\[
\left[\begin{array}{cccc}
\frac{dx_1}{dt} \\
\frac{dx_2}{dt} \\
\frac{dv_1}{dt} \\
\frac{dv_2}{dt} \\
\end{array}
\right]
=
\left[
\begin{array}{cccc}
0 & 0 & 1 & 0 \\
0 & 0 & 0 & 1 \\
-\frac{k_1}{m_1} & \frac{k_3}{m_1} & -\frac{b_1}{m_1} & 0 \\
\frac{k_3}{m_2} & -\frac{k_2}{m_2} & 0 & -\frac{b_2}{m_2} \\
\end{array}
\right]
\left[
\begin{array}{c}
x_1 \\
x_2 \\
v_1 \\
v_2 \\
\end{array}
\right]
\]
}
	\onslide<3-> {
		\item A plot of a python simulation of this system is shown in the next slide.	
}

	\end{itemize}
	\onslide<3->{\center --------------------------------------------------------------------}
	
\end{frame}

\begin{frame}
	
	\onslide<1->{
		\frametitle{Coupled ODEs} 
		\framesubtitle{Coupled Damped Harmonic Oscillator}  
	}
	\begin{center}
	\includegraphics[width=.9\textwidth,height=.6\textheight]{\dayonesegfour/coupled_damposc.png}
\end{center}
	\onslide<2->{\center --------------------------------------------------------------------}
	
\end{frame}
