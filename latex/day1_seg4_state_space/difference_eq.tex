\begin{frame}
	\frametitle{Deterministic ARMA(2, 3) Example with Stock Prices}
	
	In this example, we'll explore a deterministic ARMA(2, 3) model applied to stock prices.
	
	\begin{itemize}
		\item Stock prices often exhibit patterns and dependencies over time.
		\item We'll use an ARMA model to capture these patterns without relying on random noise.
	\end{itemize}
	
\end{frame}

\begin{frame}
	\frametitle{ARMA(2, 3) Difference Equation for Stock Prices}
	
	The deterministic ARMA(2, 3) model for stock prices is given by:
	
	\[
	X_t = 0.5 X_{t-1} + 0.2 X_{t-2} - 0.3 X_{t-3} + 0.4 X_{t-4} - 0.1 X_{t-5}
	\]
	
	Where:
	\begin{align*}
	X_t & : \text{Stock price at time } t \\
	\phi_1 = 0.5 & : \text{First autoregressive coefficient} \\
	\phi_2 = 0.2 & : \text{Second autoregressive coefficient} \\
	\theta_1 = -0.3 & : \text{First moving average coefficient} \\
	\theta_2 = 0.4 & : \text{Second moving average coefficient} \\
	\theta_3 = -0.1 & : \text{Third moving average coefficient}
	\end{align*}
	
\end{frame}

\begin{frame}
	\frametitle{Interpreting the Model for Stock Prices}
	
	In this deterministic ARMA(2, 3) model for stock prices:
	\begin{itemize}
		\item The current stock price $X_t$ depends on its own past values, $X_{t-1}$ and $X_{t-2}$, with positive coefficients ($\phi_1$ and $\phi_2$).
		\item The model also considers the past three price changes ($X_{t-3}$, $X_{t-4}$, and $X_{t-5}$) with negative coefficients ($\theta_1$, $\theta_2$, and $\theta_3$).
	\end{itemize}
	
	This model allows us to capture patterns and dependencies in stock prices over time.
	
\end{frame}

\begin{frame}
	\frametitle{Forecasting Stock Prices with ARMA(2, 3)}
	
	Given the difference equation, you can use it to forecast future stock prices.
	
	\begin{itemize}
		\item To forecast the stock price at $t+1$, you would use the equation with $X_t$, $X_{t-1}$, and $X_{t-2}$.
		\item As time progresses, you can continue forecasting future stock prices using the same equation.
	\end{itemize}
	
	This deterministic model enables us to make predictions for stock prices based on historical patterns.
	
\end{frame}

	
	\begin{frame}
		\frametitle{Introduction to ARMA}
		
		\begin{itemize}
			\item ARMA stands for AutoRegressive Moving Average.
			\item It's a commonly used statistical model for time series data.
			\item ARMA models can help us understand and forecast time-dependent data.
			\item They are a combination of autoregressive (AR) and moving average (MA) components.
		\end{itemize}
		
	\end{frame}
	
	\begin{frame}
		\frametitle{ARMA(p, q) Difference Equation}
		
		The general form of an ARMA(p, q) model is given by:
		
		\[
		X_t = c + \phi_1 X_{t-1} + \phi_2 X_{t-2} + \ldots + \phi_p X_{t-p} + \varepsilon_t + \theta_1 \varepsilon_{t-1} + \theta_2 \varepsilon_{t-2} + \ldots + \theta_q \varepsilon_{t-q}
		\]
		
		Where:
		\begin{align*}
		X_t & : \text{The observed value at time } t \\
		c & : \text{A constant or intercept term} \\
		\phi_1, \phi_2, \ldots, \phi_p & : \text{AR coefficients} \\
		\varepsilon_t & : \text{White noise or error term at time } t \\
		\theta_1, \theta_2, \ldots, \theta_q & : \text{MA coefficients}
		\end{align*}
		
	\end{frame}
	
	\begin{frame}
		\frametitle{Example: ARMA(2, 1)}
		
		Consider an ARMA(2, 1) model:
		
		\[
		X_t = c + \phi_1 X_{t-1} + \phi_2 X_{t-2} + \varepsilon_t + \theta_1 \varepsilon_{t-1}
		\]
		
		Let's say we have daily stock prices, and we want to model and forecast them using this ARMA model.
		
		\begin{itemize}
			\item $X_t$ is the stock price on day $t$.
			\item $c$ is a constant representing the long-term average price.
			\item $\phi_1$ and $\phi_2$ are autoregressive coefficients that capture the relationship between today's and previous days' prices.
			\item $\varepsilon_t$ is a white noise term representing random fluctuations.
			\item $\theta_1$ is a moving average coefficient that accounts for past prediction errors.
		\end{itemize}
		
	\end{frame}
	
	\begin{frame}
		\frametitle{Estimation and Forecasting}
		
		To estimate the ARMA model parameters ($c$, $\phi_1$, $\phi_2$, $\theta_1$), you can use statistical techniques like maximum likelihood estimation.
		
		Once the model is estimated, you can use it for forecasting future values of the time series, making it a valuable tool for time series analysis and prediction.
		
	\end{frame}
	

\begin{frame}
  \onslide<1->{\frametitle{Ordinary Difference Equations} }
\begin{itemize}
  \onslide<2->{
\item Auto-Regressive Moving Average (ARMA)  
}
  \onslide<3->{
\item  $n^{\text{th}}$ derivative corresponds to $n$-step advance: 
}
  \onslide<4->{
\begin{equation}\label{armaeq}
\armaeq
\end{equation}
}
  \onslide<5->{
\item The left side of Eq. \eqref{armaeq} is the \textbf{Auto-Regressive} part  
}
  \onslide<6->{
\item The right side of Eq. \eqref{armaeq} is the \textbf{Moving Average} part  
}
\end{itemize}
\onslide<7->{\center --------------------------------------------------------------------}
\end{frame}


\begin{frame}
  \onslide<1->{\frametitle{Ordinary Difference Equations} 
\framesubtitle{Example}
}
\begin{itemize}
  \onslide<2->{
\item Let $n=3$, $m=2$ and consider the difference equation:
}
  \onslide<3->{
\begin{multline}\label{differenceexample}
y(k+3)+1.5y(k+2)+7y(k+1)-3.2y(k)\\
=23u(k+2)-5u(k+1)+0.5u(k)
\end{multline}  
}
  \onslide<4->{
  \item A state space model can now be derived using the companion matrix method exactly as for differential equations
}
\end{itemize}
\onslide<5->{\center --------------------------------------------------------------------}
\end{frame}

\begin{frame}
\framesubtitle{Example}
  \onslide<1->{\frametitle{Ordinary Difference Equations} }
\begin{itemize}
  \onslide<2->{
\item  The homgenerous version of the difference equation \eqref{differenceexample} can be converted as follows:
}
  \onslide<3->{
\item First put all but the highest order $y$'s on the right side:\\
$y(k+3) = 3.2y(k) -7y(k+1) - 1.5(k+2)$
}
  \onslide<4->{
\item Define the $x$'s:  $x_1(k)=y(k), x_2(k)=y(k+1), x_3(k)=y(k+2)$
}
  \onslide<5->{
\item Then the state space model is:\\
$
\begin{bmatrix}
x_1(k+1)\\
x_2(k+1)\\
x_3(k+1) 
\end{bmatrix}
=
\begin{bmatrix}
0&1&0\\
0&0&1\\
3.2&-7&-1.5 
\end{bmatrix}
\begin{bmatrix}
x_1(k)\\
x_2(k)\\
x_3(k) 
\end{bmatrix}$
}
\end{itemize}
  \onslide<6->{
\begin{tiny}The insertion of the $u$'s as a matrix-vector product is left as an exercise. \end{tiny}
}
\onslide<7->{\center --------------------------------------------------------------------}
\end{frame}


% stock model with volume

\begin{frame}
	\frametitle{ARMA Model with Volume Influence on Stock Prices}
	\framesubtitle{Introduction}
	
	\begin{itemize}
		\onslide<2->{
			\item In this presentation, we'll explore an ARMA (AutoRegressive Moving Average) model for stock prices that accounts for the impact of trading volume.
		}
		\onslide<3->{
			\item We'll consider how trading volume can act as a retarding force on the direction of stock price movement.
		}
	\end{itemize}
\end{frame}

\begin{frame}
	\frametitle{ARMA Model Overview}
	\framesubtitle{Basic Concepts}
	
	\begin{itemize}
		\onslide<2->{
			\item ARMA models are used to describe time series data, such as stock prices.
		}
		\onslide<3->{
			\item They consist of two main components: AutoRegressive (AR) and Moving Average (MA) terms.
		}
		\onslide<4->{
			\item The AR component models the stock's dependence on its past values.
		}
		\onslide<5->{
			\item The MA component models the influence of past errors (residuals) on the current value.
		}
	\end{itemize}
\end{frame}


\begin{frame}
	\frametitle{Incorporating Volume}
	\framesubtitle{Volume as a Retarding Force}
	
	\begin{itemize}
		\onslide<2->{
			\item To account for volume's influence, we introduce it as a stochastic variable.
		}
		\onslide<3->{
			\item Trading volume can be used to determine the retarding effect on stock price movements.
		}
		\onslide<4->{
			\item When volume is increasing while the stock is trending down, it may slow or even reverse the trend.
		}
	\end{itemize}
\end{frame}


\begin{frame}
	\frametitle{ARMA Model with Volume}
	\framesubtitle{Equation Setup}
	
	\begin{itemize}
		\onslide<2->{
			\item We extend the traditional ARMA model with an additional term that accounts for trading volume.
		}
		\onslide<3->{
			\item The volume term introduces a retarding effect:
		}
		\onslide<4->{
			\[
			Y_t = \phi_1 Y_{t-1} + \phi_2 Y_{t-2} + \theta_1 \varepsilon_{t-1} + \theta_2 \varepsilon_{t-2} + \eta V_t
			\]
		}
		\onslide<5->{
			Where:
			\begin{itemize}
				\item \(Y_t\) is the stock price at time \(t\).
				\item \(\varepsilon_{t-1}\) and \(\varepsilon_{t-2}\) are lagged error terms.
				\item \(V_t\) is the trading volume at time \(t\).
				\item \(\phi_1, \phi_2, \theta_1, \theta_2, \eta\) are model parameters.
			\end{itemize}
		}
	\end{itemize}
\end{frame}


\begin{frame}
	\frametitle{Volume Impact}
	\framesubtitle{Interpreting Volume Influence}
	
	\begin{itemize}
		\onslide<2->{
			\item When ($\eta$) is positive, an increase in trading volume can have a retarding effect on the stock price.
		}
		\onslide<3->{
			\item This means that a rising stock price may slow down or even reverse when trading volume increases, reflecting possible market sentiment changes.
		}
		\onslide<4->{
			\item The model allows us to capture these dynamics.
		}
	\end{itemize}

\end{frame}
	
	
\begin{frame}
	\frametitle{Conclusion}
	\framesubtitle{ARMA Stock Price Model with Volume}
	
	\begin{itemize}
		\onslide<2->{
			\item The ARMA model with volume influence provides a framework to incorporate trading volume as a retarding force on stock price movements.
		}
		\onslide<3->{
			\item This model can be used to capture complex dynamics in financial markets and enhance forecasting accuracy.
		}
		\onslide<4->{
			\item Further research and data analysis are often necessary to estimate model parameters effectively.
		}
	\end{itemize}
\end{frame}
	


% stock price volume as predator-prey but discrete difference eq.	

\begin{frame}
	\frametitle{Discrete State Vector Model for Stock Prices with Volume Influence}
	\framesubtitle{Introduction}
	
	\begin{itemize}
		\onslide<2->{
			\item In this presentation, we explore a discrete state vector model for stock prices that accounts for the interplay between trading volume and price dynamics.
		}
		\onslide<3->{
			\item The model takes a predator-prey form in discrete difference equations, reflecting the complex relationship between volume and price.
		}
	\end{itemize}
	\onslide<4->{\center --------------------------------------------------------------------}
\end{frame}

\begin{frame}
	\frametitle{Model Overview}
	\framesubtitle{Basic Concepts}
	
	\begin{itemize}
		\onslide<2->{
			\item Our discrete state vector model extends the previous ARMA framework by introducing discrete difference equations for stock price (\(P\)) and trading volume (\(V\)).
		}
		\onslide<3->{
			\item These equations capture the predator-prey relationship, depicting how volume and price influence each other.
		}
		\onslide<4->{
			\item The model reveals the complex interactions between trading activity and stock price movements in discrete time steps.
		}
	\end{itemize}
	\onslide<5->{\center --------------------------------------------------------------------}
\end{frame}


\begin{frame}
	\frametitle{Difference Equations}
	\framesubtitle{Predator-Prey Relationship}
	
	\begin{itemize}
		\onslide<2->{
			\item The discrete state vector model consists of difference equations for stock price (\(P\)) and trading volume (\(V\)).
		}
		\onslide<3->{
			\item The predator-prey relationship in discrete form:
		}
		\onslide<4->{
			\[
			\begin{aligned}
			P_t &= \alpha P_{t-1} - \beta P_{t-1} V_{t-1} \\
			V_t &= \gamma P_{t-1} V_{t-1} - \delta V_{t-1}
			\end{aligned}
			\]
		}
		\onslide<5->{
			Where \(\alpha, \beta, \gamma, \delta\) are model parameters, and \(t\) represents discrete time steps.
		}
	\end{itemize}
	\onslide<6->{\center --------------------------------------------------------------------}
\end{frame}


\begin{frame}
	\frametitle{Interactions}
	\framesubtitle{Volume and Price Dynamics}
	
	\begin{itemize}
		\onslide<2->{
			\item Within this model, trading volume (\(V\)) exerts a retarding effect on the growth of stock price (\(P\)).
		}
		\onslide<3->{
			\item When volume increases while the price is rising, it may slow down or even reverse the trend, mimicking a predator-prey relationship.
		}
		\onslide<4->{
			\item Likewise, stock price influences trading volume, creating a dynamic feedback loop.
		}
	\end{itemize}
	\onslide<5->{\center --------------------------------------------------------------------}
\end{frame}


\begin{frame}
	\frametitle{Complex Dynamics}
	\framesubtitle{Capturing Market Behavior}
	
	\begin{itemize}
		\onslide<2->{
			\item The predator-prey relationship between volume and price captures the intricate dynamics of financial markets in discrete time.
		}
		\onslide<3->{
			\item Market sentiment, trading activity, and price movements are intertwined in this model, revealing complex market behavior.
		}
		\onslide<4->{
			\item This approach enhances our understanding of market dynamics and can improve forecasting accuracy.
		}
	\end{itemize}
	\onslide<5->{\center --------------------------------------------------------------------}
\end{frame}


\begin{frame}
	\frametitle{Conclusion}
	\framesubtitle{}
	
	\begin{itemize}
		\onslide<2->{
			\item The discrete state vector model with a predator-prey relationship between trading volume and stock price offers a powerful tool for analyzing financial market dynamics in discrete time steps.
		}
		\onslide<3->{
			\item This model captures the complex interplay between volume and price, providing insights into real-world financial market behavior.
		}
		\onslide<4->{
			\item It can be valuable for risk assessment, forecasting, and understanding market dynamics in discrete time.
		}
	\end{itemize}
	\onslide<5->{\center --------------------------------------------------------------------}
\end{frame}



	
