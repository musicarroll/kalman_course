\begin{frame}
	\onslide<1->{
		\frametitle{State Space Models}
		\framesubtitle{Satellite Orbits}
	}
	\begin{itemize}
		\onslide<2->{
			\item 	Simplified model of a circular orbit about a perfectly spherical planet\\  (Note: Earth is an oblate spheroid, not a sphere!)	
		}
		\onslide<3->{
			\item     The state vector consists of position and velocity components in the Planet-Centered Inertial (PCI) frame:
		\[
		\mathbf{x} = \begin{bmatrix}
		x \\ y \\ z \\ \dot{x} \\ \dot{y} \\ \dot{z}
		\end{bmatrix}
		\]
		}
	\end{itemize}		
\end{frame}

\begin{frame}
	\frametitle{State Space Models}
\framesubtitle{Satellite Orbits}
	The dynamics of the satellite are governed by the following differential equations:
	\begin{align*}
	\dot{x} &= \dot{u} \\
	\dot{y} &= \dot{v} \\
	\dot{z} &= \dot{w} \\
	\end{align*}
\end{frame}

\begin{frame}
	\frametitle{State Space Models}
\framesubtitle{Satellite Orbits}
	\begin{align*}
	\ddot{x} &= -\frac{GM}{r^3}x \\
	\ddot{y} &= -\frac{GM}{r^3}y \\
	\ddot{z} &= -\frac{GM}{r^3}z
	\end{align*}
	Where:
	\begin{itemize}
		\item \(G\) is the gravitational constant.
		\item \(M\) is the mass of the planet.
		\item \(r = \sqrt{x^2 + y^2 + z^2}\) is the distance from the satellite to the center of the planet.
	\end{itemize}
\end{frame}
