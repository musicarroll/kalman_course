

\begin{frame}
  \onslide<1->{\frametitle{Dynamic Systems} 
  \framesubtitle{Differential Equations and the Concept of State Space}
}
  
\begin{itemize}
  \onslide<2->{
\item Time-dependent behavior of physical systems often
 modeled analytically by \emph{differential equations}
}
  \onslide<3->{
\item  Called \textit{\textbf{dynamic systems}} or
 \textit{\textbf{dynamical systems}}
}
  \onslide<4->{
	\item  Smooth processes:  described by continuous, differentiable functions of time
}
\end{itemize}
\onslide<5->{\center --------------------------------------------------------------------}
\end{frame}


\begin{frame}
	\onslide<1->{\frametitle{Dynamic Systems}
	\framesubtitle{Newton's Laws of Motion} }
	\begin{itemize}
		\onslide<2->{
			\item Newton's first two laws specify what happens when a body is subjected to a force.   
		}
		\onslide<3->{
			\item Whenever a net force acts on a body, the body's velocity will change, i.e., it will accelerate or decelerate  (First Law)
		}
		\onslide<4->{
			\item For a constant net force, the motion will be such that the product of the body's mass and its acceleration will be invariant (i.e., the same for all bodies) (Second Law)
		}
		\begin{itemize}
		\onslide<5->{
			\item Under the same constant force, large bodies will exhibit a small acceleration, while small bodies will exhibit a larger acceleration
		}
		\end{itemize}
	\end{itemize}
	\onslide<6->{\center --------------------------------------------------------------------}
\end{frame}



\begin{frame}
  \onslide<1->{\frametitle{Dynamic Systems}
  \framesubtitle{Newton's Laws of Motion}
 }
\begin{itemize}
  \onslide<2->{
\item In Newton's Second law $F=ma$, the resultant force $F$ acts on the mass $m$ to produce an acceleration:
\begin{equation}
F=ma=m\dot{v}=m\deriv{v}{t}=m\secondderiv{y}{t}=m\ddot{y}
\end{equation}  
}
  \onslide<3->{
	\item Since acceleration is the second derivative of position, Newton's Second law often leads to \textbf{second order} differential equations (whenever the acceleration is constant, i.e., $\dot{a}=0$); the solution $y(t)$ is considered to be the \textbf{output} of the dynamical system   
}
  \onslide<4->{
	\item The resultant force is the vector sum of all forces acting on the body, both \textbf{internal} coupling forces and \textbf{external} input forces  
}
\end{itemize}
\onslide<5->{\center --------------------------------------------------------------------}
\end{frame}

\begin{frame}
	\onslide<1->{\frametitle{Dynamic Systems} 
	\framesubtitle{A Note on Notation}
	}
	\begin{itemize}
		\onslide<2->{ \item In pure mathematics, one often sees $y$ as a function of $x$:  $y=f(x)$.  
		}
		\onslide<3->{
			\item  However, in engineering work, the variable $x$ is often reserved to be used as the (vector) state variable.
		}
		\onslide<4->{
			\item Moreover, in dynamic systems, $y$ is generally considered to be a function of time $t$.  Hence our use of symbols such as \\ $y(t)$, $\dfrac{dy}{dt}$, $\Dopnth{n}{t}{y}$, $\dot{y}$, $\ddot{y}$, etc.
		}
		\onslide<5->{
			\item  In physics, the variable $r=r(t)$ is often used to stand for displacement (position), with 3d coordinates of $x,y,z$.
		}
		\onslide<6->{
			\item In these slides we are mostly using standard engineering notation.
		}
		\onslide<7->{
			\item It is also common in engineering notation to use $u=u(t)$ as the symbol for external input -- or control -- function.
		}
	\end{itemize}
	\onslide<8->{\center --------------------------------------------------------------------}
\end{frame}

\begin{frame}
  \onslide<1->{\frametitle{Dynamic Systems} }
\begin{itemize}
  \onslide<2->{ \item Which forces are considered internal and which external depends on how the system boundary is drawn  
}
  \onslide<3->{
\item  Simple harmonic oscillator has one internal force determining system behavior: $F=-ky$, because the spring itself is often considered a (massless) part of the system  \\(more complicated oscillators may also include internal frictional and/or external driving forces)
}
\onslide<4->{
	\item Spring forces (e.g., Hooke's law, where $F=-ky$) are often considered to be \textbf{internal} to system
}
  \onslide<5->{
\item  External forces (e.g., constant or time-dependent forcing functions) may also be present, e.g.:
 \[\frac{d^2y}{dt^2}+\frac{k}{m}y=\dfrac{1}{m}\sin\omega t\]
}
\onslide<6->{
	\item Internal forces are generally not viewed as \textbf{inputs}
}
\end{itemize}
\onslide<7->{\center --------------------------------------------------------------------}
\end{frame}

\begin{frame}
  \onslide<1->{\frametitle{Dynamic Systems} }
\begin{itemize}
  \onslide<2->{
\item Most general second-order equation for a mechanical
 system of this type (according to Newton's law) is:
\begin{equation}\label{second ord}
 \ddot{y} =\frac{1}{m}F(y,\dot{y},t)
\end{equation}  
}
  \onslide<3->{
\item If $F(y,\dot{y},t)$ can be expressed 
as sum of $F_1(y,\dot{y})$ and $F_2(t)$, then
 we can write this in input-output form:
  \[\ddot{y}-\frac{1}{m}F_1(y,\dot{y})=\frac{1}{m}F_2(t)\]
}
  \onslide<4->{
	\item Note that the left hand side is a function only of $y$ and its derivatives, while the right hand side is a function of time only.
}
\end{itemize}
\onslide<5->{\center --------------------------------------------------------------------}
\end{frame}


\begin{frame}
  \onslide<1->{
  	\frametitle{Dynamic Systems}
	\framesubtitle{Measurement Models}  
 }
\begin{itemize}
  \onslide<2->{
	\item  In engineering applications, we often model not
	 only system dynamics but also the processes by which we \textbf{measure} system outputs
	}
	\begin{itemize}
		\onslide<3->{\item This is usually because we need to control the system dynamics
		}
		\onslide<4->{\item But more often than not we need to estimate certain quantities of interest associated with our system
		}
		\onslide<5->{\item And we can't control or change the behavior of something we can't accurately estimate!
		}
	\end{itemize}
\end{itemize}
\onslide<6->{\center --------------------------------------------------------------------}
\end{frame}

\begin{frame}
	\onslide<1->{
		\frametitle{Dynamic Systems}
		\framesubtitle{Measurement Models}  
	}
	\begin{itemize}
		\onslide<2->{
			\item  But outputs are not always directly observable!! They might have a complicated relationship to something we can physically measure
		}
		\onslide<3->{
			\item Thus, in modeling dynamic systems, we often add a second set of 
			equations, usually algebraic or functional in nature, that relates
			the output $y$ (and possibly the input $u$) to some measured or observed quantities
			\[z = h(y)+g(u)\]  
		}
	\end{itemize}
	\onslide<4->{\center --------------------------------------------------------------------}
\end{frame}

\begin{frame}
	\onslide<1->{\frametitle{The Concept of State} }
	\begin{itemize}
		\onslide<2->{
			\item Concept of state involves in some sense the instantaneous internal arrangement and relationships of a system's components
		}
		\onslide<3->{
			\item In practical  situations, the state variables may represent attributes of the  system that we wish to know something about 
		}
		\onslide<4->{
			\item Also intuitively, state variables are usually naturally coupled in some sense; 
			this is obvious when they are derived from a single differential equation in which all quantities are derivatives 
			of a single variable (the output variable)  
		}
	\end{itemize}
	\onslide<5->{\center --------------------------------------------------------------------}
\end{frame}


\begin{frame}
  \onslide<1->{\frametitle{The Concept of State} 
	\framesubtitle{State as Initial Conditions}  
}
\begin{itemize}
  \onslide<2->{
	\item \footnotesize{"How is a dynamic system (linear or nonlinear) described?
		The fundamental concept is the notion of the state.  By this is meant, intuitively, some quantitative information 
		(a set of numbers, a function, etc.) which is the least amount of data one has to know about the past behavior of 
		the system in order to predict its future behavior."\\
		Rudolf Kalman, in seminal paper (1960)}
}
  \onslide<3->{
\item To determine behavior of deterministic system at any future time requires knowledge of two things:
\begin{enumerate}
\item Equations governing the system dynamics (usually differential of difference equations)
\item Initial conditions
\end{enumerate}  
}
  \onslide<4->{
\item But initial conditions for an $n^{\text{th}}$-order continuous dynamic system are simply values of first $n$ derivatives at initial time (the Identity operator can be considered the 0th order derivative)
}
  \onslide<5->{
	\item The $n^{\text{th}}$-order derivative is given essentially by the differential equation itself
}

\end{itemize}
\onslide<5->{\center --------------------------------------------------------------------}
\end{frame}

\begin{frame}
  \onslide<1->{\frametitle{The Concept of State} }
\begin{itemize}
  \onslide<2->{
\item At any point in time, we can think of instantaneous values of first $n$ derivatives as initial conditions for all future dynamic evolution  
}
  \onslide<3->{
\item The first $n$ derivatives are the $0^\text{th},1^\text{st},\ldots,(n-1)^\text{st}$ derivatives of the output function $y$
}
  \onslide<4->{
	\item These values capture the "state" of the dynamic system at any time after the time of the initial conditions  
}
\end{itemize}
\onslide<4->{\center --------------------------------------------------------------------}
\end{frame}

\begin{frame}
  \onslide<1->{\frametitle{The Concept of State} 
	\framesubtitle{Relation to Taylor's Series}  
}
\begin{itemize}
  \onslide<2->{
\item  Concept of "state" also encoded in Taylor series of a function 
}
  \onslide<3->{
\item Taylor's theorem: if a function $y(t)$ has derivatives of all orders near $t_0$, then its value at any point $t$ near $t_0$ can be determined from the values of its derivatives at $t_0$:
}
  \onslide<4->{
\begin{equation}
\taylor{y}{t}{t_0}
\end{equation}
}
  \onslide<5->{
	\item I.e., the value of the function at a time near $t_0$ is completely determined by its value and the values of all of its derivatives at $t_0$\\ (Note that this works for time points in the past as well as those in the future, provided they are sufficiently close to $t_0$)
}
\end{itemize}
\onslide<6->{\center --------------------------------------------------------------------}
\end{frame}

\begin{frame}
  \onslide<1->{\frametitle{The Concept of State} 
	\framesubtitle{Relation to Taylor's Series}  
}
\begin{itemize}
  \onslide<2->{
\item In the language of dynamic systems, the quantity $t-t_0$\\ (if $>0$) is simply the time elapsed since the initial time $t_0$ (otherwise, it is the negative duration from $t_0$ to a past time point $t$).  
}
  \onslide<3->{
\item Thus, if you have knowledge of the state at time $t_0$, i.e., if you know the values of all the (infinitely many) derivatives, then you can determine the value of the function at any later time (sufficiently close to $t_0$).  
}
  \onslide<4->{
\item  This is because all of the dynamic information is contained in the derivatives.
}
\end{itemize}
\onslide<5->{\center --------------------------------------------------------------------}
\end{frame}

\begin{frame}
  \onslide<1->{\frametitle{Concept of State} 
	\framesubtitle{Relation to Taylor's Series}  
  
}
  \onslide<2-> {
	If $y(t)$ is smooth, each of its derivatives also has a Taylor series:
}
  \onslide<3-> {
	\footnotesize{\begin{multline}
	\left[\begin{array}{c}
	y(t) \\
	\Dopnth{1}{t}{y}(t) \\
	\vdots \\
	\Dopnth{n}{t}{y}(t) \\
	\end{array}\right] \approx\\
	\Taylormats{n}{t}{y}{n}{(t-t_0)}
	\end{multline}}
}
\onslide<4->{\center --------------------------------------------------------------------}
\end{frame}


\begin{frame}
  \onslide<1->{\frametitle{The Concept of State} }
\begin{itemize}
  \onslide<2->{
\item  Mechanical systems like a moving particle have three simultaneous attributes:  position, velocity, and
 acceleration. 
}
  \onslide<3->{
\item Newton tells us that acceleration is often a function
 of position, velocity and perhaps time.
}
  \onslide<4->{
\item However, as we just saw, a force that acts on a system as a function of
 time only should be considered an external forcing function,
 i.e., an input to the system.  
}
  \onslide<5 -> {
\item Thus we are led to think of position and velocity as the
primary state variables of interest in mechanical systems.  
}
\end{itemize}
\onslide<5->{\center --------------------------------------------------------------------}
\end{frame}

\begin{frame}
  \onslide<1->{\frametitle{State Space Representation} }
\begin{itemize}
  \onslide<2->{
\item  A general $n$-th order linear differential equation can be expressed as a system of first-order differential equations.
}
  \onslide<3->{
\item Ignoring the input $u$ for the moment, consider:
 \[p(D)y = D^ny+a_{n-1}D^{n-1}y+\cdots +a_1Dy+a_0y,\]
 where $D$ is the differential operator (suppressing the $t$); we have put the operator polynomial in normal form, i.e., such that its leading coefficient is 1 (e.g., dividing through by $m$ in $F=ma$).  
}
\end{itemize}
\onslide<4->{\center --------------------------------------------------------------------}
\end{frame}

\begin{frame}
  \onslide<1->{\frametitle{State Space Representation} }
\begin{itemize}
  \onslide<2->{
\item We define new variables, setting $x_1=y,x_2=Dy,\ldots$:
\begin{align*}
  x_1(t) & = y(t) \\
  x_2(t) & = Dy(t) \\
    \cdots \\
  x_n(t) & = D^{n-1}y(t).
\end{align*}  
}
  \onslide<3->{
\item The functions $x_i(t)$ are called the \textbf{states} or \textbf{state variables} of the system    
}
  \onslide<4->{
	\item These variables are merely definitions, but the derivative of $n^{\text{th}}$ variable will in fact deliver the essential characteristics of the system, since it will be the differential equation itself.
}
\end{itemize}
\onslide<5->{\center --------------------------------------------------------------------}
\end{frame}

\begin{frame}
  \onslide<1->{\frametitle{State Space Representation} }
\begin{itemize}
  \onslide<2->{
\item  Taking the derivative of these $n$ state variables and making use of the
 original differential equation yields:
\begin{align*}
  Dx_1(t) & = x_2(t) \\
  Dx_2(t) & = x_3(t) \\
    \cdots \\
  Dx_n(t) & =-a_0x_1(t)-a_1x_2(t)-\cdots -a_{n-1}x_n(t) .
\end{align*} 
}
  \onslide<3->{
\item  These $n$ equations can now be written more compactly in vector matrix
 form:  $Dx(t)=Ax(t)$ or $\dot{x}(t)=Ax(t)$. 
}
\end{itemize}
\onslide<4->{\center --------------------------------------------------------------------}
\end{frame}

\begin{frame}
  \onslide<1->{\frametitle{State Space Representation} }
\begin{itemize}
  \onslide<2->{
\item  In the equation $\dot{x}(t)=Ax(t)$, we have:
 \[x(t)=\begin{bmatrix}x_1(t)\\x_2(t)\\\vdots \\
 x_n(t)\end{bmatrix}\]
 and
 \[A=\begin{bmatrix}
 0 & 1 & 0 &\cdots & 0\\
 0 & 0 & 1 & \cdots & 0\\
 \vdots & & \ddots & & \vdots \\
 0 & 0  & 0 & 0 & 1\\
 -a_{0} &-a_{1} & \cdots & -a_{n-2} & -a_{n-1}
 \end{bmatrix},\]
 the \textit{\textbf{companion matrix}} of the differential
 equation. 
}
\end{itemize}
\onslide<3->{\center --------------------------------------------------------------------}
\end{frame}

\begin{frame}
  \onslide<1->{\frametitle{State Space Representation} }
\begin{itemize}
  \onslide<2->{
\item If there is an input function $u$ and it is present via its
 derivatives, then we can write:
\begin{equation}\label{statedyn}
  \dot{x}=Ax+Bu
\end{equation}
forming $B$ in much the same way we formed $A$, and suppressing
the time variable $t$.   
}
  \onslide<3->{
\item  Equation \eqref{statedyn} expresses the system's \textit{\textbf{state dynamics}}.
 \item Later we will include white noise as an input.  However,
 that will make the differential equations stochastic, and that will require a deeper dive into stochastic processes. 
}
\end{itemize}
\onslide<4->{\center --------------------------------------------------------------------}
\end{frame}

\begin{frame}
  \onslide<1->{\frametitle{State Space Representation} }
\begin{itemize}
  \onslide<2->{
\item To obtain the original output $y$ as a function of the internal state of the system, we require one
 additional vector-matrix equation:
\begin{equation}\label{measdyn1}
  y = Cx + Du
\end{equation}
where the matrix $D$ (not to be confused with the differential operator!) represents a \textbf{\textit{direct feedthrough}} of the input to the output.    
}
  \onslide<3->{
\item  The output equation is derived directly from the $b_i$ coefficients in the original matrix differential equation.
}
\end{itemize}
\onslide<4->{\center --------------------------------------------------------------------}
\end{frame}

\begin{frame}
  \onslide<1->{\frametitle{State Space Representation} }
\begin{itemize}
  \onslide<2->{
\item  Kalman filter theory often assumes the matrix $D$ in the output equation is 0, and uses a measurement model instead:
\begin{equation}\label{measdyn2}
  z = Hx
\end{equation} 
}
  \onslide<3->{
\item Note that $H$ implicitly includes $C$ as a factor.  
}
  \onslide<4->{
\item Note also that we have not placed any restrictions on the size of
 the measurement column vector $z$.  As long as $H$ has $n$ columns it can have as many rows as desired.  
}
\end{itemize}
\onslide<5->{\center --------------------------------------------------------------------}
\end{frame}

\begin{frame}
  \onslide<1->{\frametitle{State Space Representation: Example} }
\begin{itemize}
  \onslide<2->{
\item Consider again the simple harmonic oscillator (suppressing $t$):
 \[m\frac{d^2y}{dt^2}+ky=0\]
 and convert this to state space form:
\begin{align*}
  x_1 & =y \\
  x_2 & =Dy
\end{align*}
and $\dot{x}_2 = - \frac{k}{m}x_1 - 0\cdot x_2 $.
}
  \onslide<3->{
\item Therefore, in matrix and vector form:
 \[
 \begin{bmatrix}
 \dot{x}_1\\
 \dot{x}_2
 \end{bmatrix}
 =
\begin{bmatrix}
 0 & 1 \\
 -k/m & 0
 \end{bmatrix}
 \begin{bmatrix}
 x_1\\
 x_2
 \end{bmatrix}.
 \]
}
\end{itemize}
\onslide<4->{\center --------------------------------------------------------------------}
\end{frame}

\begin{frame}
  \onslide<1->{\frametitle{State Space Representation: Example} }
\begin{itemize}
  \onslide<2->{
\item  For a measurement model, we could consider a camera with a strobe flash that periodically records the instantaneous position of the sliding mass.
}
  \onslide<3->{
\item   In this case, only the position would be observable and we would write the measurement or observation model as:
 \[
 z =
\begin{bmatrix}
 1 & 0
 \end{bmatrix}
 \begin{bmatrix}
 x_1\\
 x_2
 \end{bmatrix}.
 \]
(Remember that we are ignoring both process and measurement noise.)   
}
\end{itemize}
\onslide<4->{\center --------------------------------------------------------------------}
\end{frame}

% Add Python package overview

\begin{frame}
  \onslide<1->{\frametitle{State Space Representation} }
\begin{itemize}
  \onslide<2->{
\item A single $n^\text{th}$ order ordinary differential equation presents only one way of deriving a state space representation for a dynamic system 
}
  \onslide<3->{
\item Can also derive state space models directly from systems of coupled ODEs
}
  \onslide<4->{
\begin{itemize}
\item Real world systems consisting of electrical, mechanical, thermal, hydraulic components are coupled in this way:  e.g., a DC motor driving an inertial load
\end{itemize}
}
  \onslide<5->{
\item In discrete time, ordinary difference equations also lead to state space models
}
\end{itemize}
\onslide<6->{\center --------------------------------------------------------------------}
\end{frame}

% coupled ODEs

% coupled_damposcs.tex

\begin{frame}
	
	\onslide<1->{
		\frametitle{Coupled ODEs} 
		\framesubtitle{Coupled Damped Harmonic Oscillator}  
	}
	\begin{itemize}
		\onslide<2->{
			\item Two damped harmonic oscillators can be coupled via a third spring between the masses.
		}
		\onslide<3->{
			\item Consider two masses, $m_1$ and $m_2$, attached to springs with spring constants $k_1$ and $k_2$, respectively. 
		}
		\onslide<4->{
			\begin{itemize}
				\item The two masses are connected by a coupling spring with a spring constant $k_3$. The damping coefficients for the two masses are denoted as $b_1$ and $b_2$, respectively.
			\end{itemize}
		}
		\onslide<5->{
			\item A schematic drawing of this system is depicted on the next slide 
		}
	\end{itemize}
	\onslide<6->{\center --------------------------------------------------------------------}
	
\end{frame}



\begin{frame}
	
	\onslide<1->{
		\frametitle{Coupled ODEs} 
		\framesubtitle{Coupled Damped Harmonic Oscillator}  
	}
	\begin{center}
		\includegraphics[width=.9\textwidth,height=.6\textheight]{\dayonesegfour/coupled_damped_harmosc.png}
	\end{center}
	
	\onslide<2->{\center --------------------------------------------------------------------}
	
\end{frame}

\begin{frame}
	
	\onslide<1->{
		\frametitle{Coupled ODEs} 
		\framesubtitle{Coupled Damped Harmonic Oscillator}  
	}
	\begin{itemize}
		\onslide<2->{
			\item We can define the state variables as follows:
			\begin{itemize}
				\item $x_1$ = displacement of mass $m_1$ from its equilibrium position
				\item $x_2$ = displacement of mass $m_2$ from its equilibrium position
				\item $v_1$ = velocity of mass $m_1$
				\item $v_2$ = velocity of mass $m_2$
			\end{itemize}
			
		}
		\onslide<3->{
			\item The state space equations can be written as:
			
			\[
			\frac{{dx_1}}{{dt}} = v_1
			\]
			\[
			\frac{{dx_2}}{{dt}} = v_2
			\]
			\[
			\frac{{dv_1}}{{dt}} = \frac{{-k_1 \cdot x_1 - k_3 \cdot (x_1 - x_2) - b_1 \cdot v_1}}{{m_1}}
			\]
			\[
			\frac{{dv_2}}{{dt}} = \frac{{-k_2 \cdot x_2 + k_3 \cdot (x_1 - x_2) - b_2 \cdot v_2}}{{m_2}}
			\]
		}
	\end{itemize}
	\onslide<4->{\center --------------------------------------------------------------------}
	
\end{frame}


\begin{frame}
	
	\onslide<1->{
		\frametitle{Coupled ODEs} 
		\framesubtitle{Coupled Damped Harmonic Oscillator}  
	}
	\begin{itemize}
		\onslide<2->{
			\item In matrix form, the state space model can be written as:
			
\[
\left[\begin{array}{cccc}
\frac{dx_1}{dt} \\
\frac{dx_2}{dt} \\
\frac{dv_1}{dt} \\
\frac{dv_2}{dt} \\
\end{array}
\right]
=
\left[
\begin{array}{cccc}
0 & 0 & 1 & 0 \\
0 & 0 & 0 & 1 \\
-\frac{k_1}{m_1} & \frac{k_3}{m_1} & -\frac{b_1}{m_1} & 0 \\
\frac{k_3}{m_2} & -\frac{k_2}{m_2} & 0 & -\frac{b_2}{m_2} \\
\end{array}
\right]
\left[
\begin{array}{c}
x_1 \\
x_2 \\
v_1 \\
v_2 \\
\end{array}
\right]
\]
}
	\onslide<3-> {
		\item A plot of a python simulation of this system is shown in the next slide.	
}

	\end{itemize}
	\onslide<3->{\center --------------------------------------------------------------------}
	
\end{frame}

\begin{frame}
	
	\onslide<1->{
		\frametitle{Coupled ODEs} 
		\framesubtitle{Coupled Damped Harmonic Oscillator}  
	}
	\begin{center}
	\includegraphics[width=.9\textwidth,height=.6\textheight]{\dayonesegfour/coupled_damposc.png}
\end{center}
	\onslide<2->{\center --------------------------------------------------------------------}
	
\end{frame}



\begin{frame}
		\frametitle{Coupled ODEs}
		\framesubtitle{Predator-Prey State Space Model (Equations)}
		The state space equations for a Lotka-Volterra predator-prey system can be written as follows:
		
		\[
		\frac{dV}{dt} = r_v V - c_{vp} V P
		\]
		
		\[
		\frac{dP}{dt} = c_{pv} V P - d_p P
		\]
		
		Where:
		\begin{align*}
		V & : \text{Prey population} \\
		P & : \text{Predator population} \\
		r_v & : \text{Prey intrinsic growth rate} \\
		c_{vp} & : \text{Predation rate} \\
		c_{pv} & : \text{Conversion efficiency} \\
		d_p & : \text{Predator death rate}
		\end{align*}
		
\end{frame}

\begin{frame}
	\onslide<1->{
		\frametitle{State Space Representation} 
		\framesubtitle{Assumptions Underlying Predator-Prey Equations}
	}
	\begin{itemize}
		\onslide<2->{
			\item Sufficient food is always available for the prey population 
		}
		\onslide<3->{
			\item  Food supply of predator population depends only on size of prey population (predators eat only one kind of thing)
		}
		\onslide<4->{
			\item  Current rate of change of population is proportional to its current size
		}
		\onslide<5->{
			\item Environment does not change to favor one species over another
		}
		\onslide<6-> {
			\item Genetic adaptation is negligible
		}
		\onslide<7->{
			\item Predators are insatiable
		}
	\end{itemize}
	\onslide<8->{\center --------------------------------------------------------------------}
\end{frame}

	
	\begin{frame}
		\frametitle{Coupled ODEs}
		\framesubtitle{Predator-Prey State Space Model (Matrix Form)}
		In matrix form, the state space model can be expressed as:
		
		\[
		\left[\begin{array}{c}
		\frac{dV}{dt} \\
		\frac{dP}{dt}
		\end{array}\right]
		=
		\left[\begin{array}{cc}
		r_v & -c_{vp} \\
		c_{pv} & -d_p
		\end{array}\right]
		\left[\begin{array}{c}
		V \\
		P
		\end{array}\right]
		\]
		
	\end{frame}




\begin{frame}
  \onslide<1->{
	\frametitle{State Space Representation} 
	\framesubtitle{Example:  Predator-Prey Equations, Real World}
}
  \onslide<2->{
\begin{center}
\includegraphics[width=.9\textwidth,height=.6\textheight]{\dayonesegfour/predprey.png}
\end{center}
}
\onslide<3->{\center --------------------------------------------------------------------}
\end{frame}


\begin{frame}
	\onslide<1->{
		\frametitle{State Space Models}
		\framesubtitle{Satellite Orbits}
	}
	\begin{itemize}
		\onslide<2->{
			\item 	Simplified model of a circular orbit about a perfectly spherical planet\\  (Note: Earth is an oblate spheroid, not a sphere!)	
		}
		\onslide<3->{
			\item     The state vector consists of position and velocity components in the Planet-Centered Inertial (PCI) frame:
		\[
		\mathbf{x} = \begin{bmatrix}
		x \\ y \\ z \\ \dot{x} \\ \dot{y} \\ \dot{z}
		\end{bmatrix}
		\]
		}
	\end{itemize}		
\end{frame}

\begin{frame}
	\frametitle{State Space Models}
\framesubtitle{Satellite Orbits}
	The dynamics of the satellite are governed by the following differential equations:
	\begin{align*}
	\dot{x} &= \dot{u} \\
	\dot{y} &= \dot{v} \\
	\dot{z} &= \dot{w} \\
	\end{align*}
\end{frame}

\begin{frame}
	\frametitle{State Space Models}
\framesubtitle{Satellite Orbits}
	\begin{align*}
	\ddot{x} &= -\frac{GM}{r^3}x \\
	\ddot{y} &= -\frac{GM}{r^3}y \\
	\ddot{z} &= -\frac{GM}{r^3}z
	\end{align*}
	Where:
	\begin{itemize}
		\item \(G\) is the gravitational constant.
		\item \(M\) is the mass of the planet.
		\item \(r = \sqrt{x^2 + y^2 + z^2}\) is the distance from the satellite to the center of the planet.
	\end{itemize}
\end{frame}


\begin{frame}
	\onslide<1->{
		\frametitle{State Space Representation} 
		\framesubtitle{Example:  DC Motor with Inertial Load}
	}
	\begin{center}
		\includegraphics[width=.9\textwidth,height=.6\textheight]{\dayonesegfour/dc_motor_w_inertial_load.png}
	\end{center}
	\onslide<2->{\center --------------------------------------------------------------------}
\end{frame}



	
	\begin{frame}{State Space Model for a DC Motor with Inertial Load}
		
		\begin{block}{System Description}
			Consider a DC motor with an inertial load. The key variables and constants are as follows:
			\begin{align*}
			R & : \text{Motor resistance} \\
			L & : \text{Motor inductance} \\
			v_{\text{app}} & : \text{Applied voltage} \\
			v_{\text{emf}} & : \text{Back EMF} \\
			\theta & : \text{Rotational position of the load} \\
			\omega & : \text{Rate of change of }\theta \\
			K_f & : \text{Frictional constant} \\
			K_b & : \text{EMF constant} \\
			\tau & : \text{Torque} \\
			K_m & : \text{Armature constant}
			\end{align*}
		\end{block}
		
	\end{frame}
	
	\begin{frame}{State Space Equations}
		The state space equations for the DC motor with an inertial load can be written as follows:
		
		\[
		\frac{d\theta}{dt} = \omega
		\]
		
		\[
		L\frac{di}{dt} = v_{\text{app}} - v_{\text{emf}} - Ri
		\]
		
		\[
		J\frac{d\omega}{dt} = \tau - K_f\omega
		\]
		
		\[
		v_{\text{emf}} = K_b\omega
		\]
		
		Where:
		\begin{align*}
		\theta & : \text{Rotational position} \\
		\omega & : \text{Angular velocity} \\
		i & : \text{Current through the motor} \\
		J & : \text{Moment of inertia of the load} \\
		\tau & : \text{Applied torque}
		\end{align*}
		
		These equations describe the dynamics of the DC motor and the load.
	\end{frame}
	
	\begin{frame}{Matrix Form}
		In matrix form, the state space model can be expressed as:
		
		\[
		\left[\begin{array}{c}
		\frac{d\theta}{dt} \\
		\frac{di}{dt} \\
		\frac{d\omega}{dt}
		\end{array}\right]
		=
		\left[\begin{array}{ccc}
		0 & 0 & 1 \\
		0 & -\frac{R}{L} & -\frac{K_b}{L} \\
		0 & \frac{K_b}{J} & -\frac{K_f}{J}
		\end{array}\right]
		\left[\begin{array}{c}
		\theta \\
		i \\
		\omega
		\end{array}\right]
		+
		\left[\begin{array}{c}
		0 \\
		\frac{1}{L} \\
		0
		\end{array}\right]
		v_{\text{app}}
		\]
	The next slide shows the results for an applied voltage of 1.0V and various nominal parameter values.	
	\end{frame}
	

\begin{frame}
	\onslide<1->{
		\frametitle{State Space Representation} 
		\framesubtitle{Example:  DC Motor with Inertial Load}
	}
	\begin{center}
		\includegraphics[width=.9\textwidth,height=.6\textheight]{\dayonesegfour/dc_motor_w_inertial_load_plot.png}
	\end{center}
	\onslide<2->{\center --------------------------------------------------------------------}
\end{frame}


% predprey.tex
% dc_motor_w_load.tex

% difference_eq.tex
\begin{frame}
	\frametitle{Deterministic ARMA(2, 3) Example with Stock Prices}
	
	In this example, we'll explore a deterministic ARMA(2, 3) model applied to stock prices.
	
	\begin{itemize}
		\item Stock prices often exhibit patterns and dependencies over time.
		\item We'll use an ARMA model to capture these patterns without relying on random noise.
	\end{itemize}
	
\end{frame}

\begin{frame}
	\frametitle{ARMA(2, 3) Difference Equation for Stock Prices}
	
	The deterministic ARMA(2, 3) model for stock prices is given by:
	
	\[
	X_t = 0.5 X_{t-1} + 0.2 X_{t-2} - 0.3 X_{t-3} + 0.4 X_{t-4} - 0.1 X_{t-5}
	\]
	
	Where:
	\begin{align*}
	X_t & : \text{Stock price at time } t \\
	\phi_1 = 0.5 & : \text{First autoregressive coefficient} \\
	\phi_2 = 0.2 & : \text{Second autoregressive coefficient} \\
	\theta_1 = -0.3 & : \text{First moving average coefficient} \\
	\theta_2 = 0.4 & : \text{Second moving average coefficient} \\
	\theta_3 = -0.1 & : \text{Third moving average coefficient}
	\end{align*}
	
\end{frame}

\begin{frame}
	\frametitle{Interpreting the Model for Stock Prices}
	
	In this deterministic ARMA(2, 3) model for stock prices:
	\begin{itemize}
		\item The current stock price $X_t$ depends on its own past values, $X_{t-1}$ and $X_{t-2}$, with positive coefficients ($\phi_1$ and $\phi_2$).
		\item The model also considers the past three price changes ($X_{t-3}$, $X_{t-4}$, and $X_{t-5}$) with negative coefficients ($\theta_1$, $\theta_2$, and $\theta_3$).
	\end{itemize}
	
	This model allows us to capture patterns and dependencies in stock prices over time.
	
\end{frame}

\begin{frame}
	\frametitle{Forecasting Stock Prices with ARMA(2, 3)}
	
	Given the difference equation, you can use it to forecast future stock prices.
	
	\begin{itemize}
		\item To forecast the stock price at $t+1$, you would use the equation with $X_t$, $X_{t-1}$, and $X_{t-2}$.
		\item As time progresses, you can continue forecasting future stock prices using the same equation.
	\end{itemize}
	
	This deterministic model enables us to make predictions for stock prices based on historical patterns.
	
\end{frame}

	
	\begin{frame}
		\frametitle{Introduction to ARMA}
		
		\begin{itemize}
			\item ARMA stands for AutoRegressive Moving Average.
			\item It's a commonly used statistical model for time series data.
			\item ARMA models can help us understand and forecast time-dependent data.
			\item They are a combination of autoregressive (AR) and moving average (MA) components.
		\end{itemize}
		
	\end{frame}
	
	\begin{frame}
		\frametitle{ARMA(p, q) Difference Equation}
		
		The general form of an ARMA(p, q) model is given by:
		
		\[
		X_t = c + \phi_1 X_{t-1} + \phi_2 X_{t-2} + \ldots + \phi_p X_{t-p} + \varepsilon_t + \theta_1 \varepsilon_{t-1} + \theta_2 \varepsilon_{t-2} + \ldots + \theta_q \varepsilon_{t-q}
		\]
		
		Where:
		\begin{align*}
		X_t & : \text{The observed value at time } t \\
		c & : \text{A constant or intercept term} \\
		\phi_1, \phi_2, \ldots, \phi_p & : \text{AR coefficients} \\
		\varepsilon_t & : \text{White noise or error term at time } t \\
		\theta_1, \theta_2, \ldots, \theta_q & : \text{MA coefficients}
		\end{align*}
		
	\end{frame}
	
	\begin{frame}
		\frametitle{Example: ARMA(2, 1)}
		
		Consider an ARMA(2, 1) model:
		
		\[
		X_t = c + \phi_1 X_{t-1} + \phi_2 X_{t-2} + \varepsilon_t + \theta_1 \varepsilon_{t-1}
		\]
		
		Let's say we have daily stock prices, and we want to model and forecast them using this ARMA model.
		
		\begin{itemize}
			\item $X_t$ is the stock price on day $t$.
			\item $c$ is a constant representing the long-term average price.
			\item $\phi_1$ and $\phi_2$ are autoregressive coefficients that capture the relationship between today's and previous days' prices.
			\item $\varepsilon_t$ is a white noise term representing random fluctuations.
			\item $\theta_1$ is a moving average coefficient that accounts for past prediction errors.
		\end{itemize}
		
	\end{frame}
	
	\begin{frame}
		\frametitle{Estimation and Forecasting}
		
		To estimate the ARMA model parameters ($c$, $\phi_1$, $\phi_2$, $\theta_1$), you can use statistical techniques like maximum likelihood estimation.
		
		Once the model is estimated, you can use it for forecasting future values of the time series, making it a valuable tool for time series analysis and prediction.
		
	\end{frame}
	

\begin{frame}
  \onslide<1->{\frametitle{Ordinary Difference Equations} }
\begin{itemize}
  \onslide<2->{
\item Auto-Regressive Moving Average (ARMA)  
}
  \onslide<3->{
\item  $n^{\text{th}}$ derivative corresponds to $n$-step advance: 
}
  \onslide<4->{
\begin{equation}\label{armaeq}
\armaeq
\end{equation}
}
  \onslide<5->{
\item The left side of Eq. \eqref{armaeq} is the \textbf{Auto-Regressive} part  
}
  \onslide<6->{
\item The right side of Eq. \eqref{armaeq} is the \textbf{Moving Average} part  
}
\end{itemize}
\onslide<7->{\center --------------------------------------------------------------------}
\end{frame}


\begin{frame}
  \onslide<1->{\frametitle{Ordinary Difference Equations} 
\framesubtitle{Example}
}
\begin{itemize}
  \onslide<2->{
\item Let $n=3$, $m=2$ and consider the difference equation:
}
  \onslide<3->{
\begin{multline}\label{differenceexample}
y(k+3)+1.5y(k+2)+7y(k+1)-3.2y(k)\\
=23u(k+2)-5u(k+1)+0.5u(k)
\end{multline}  
}
  \onslide<4->{
  \item A state space model can now be derived using the companion matrix method exactly as for differential equations
}
\end{itemize}
\onslide<5->{\center --------------------------------------------------------------------}
\end{frame}

\begin{frame}
\framesubtitle{Example}
  \onslide<1->{\frametitle{Ordinary Difference Equations} }
\begin{itemize}
  \onslide<2->{
\item  The homgenerous version of the difference equation \eqref{differenceexample} can be converted as follows:
}
  \onslide<3->{
\item First put all but the highest order $y$'s on the right side:\\
$y(k+3) = 3.2y(k) -7y(k+1) - 1.5(k+2)$
}
  \onslide<4->{
\item Define the $x$'s:  $x_1(k)=y(k), x_2(k)=y(k+1), x_3(k)=y(k+2)$
}
  \onslide<5->{
\item Then the state space model is:\\
$
\begin{bmatrix}
x_1(k+1)\\
x_2(k+1)\\
x_3(k+1) 
\end{bmatrix}
=
\begin{bmatrix}
0&1&0\\
0&0&1\\
3.2&-7&-1.5 
\end{bmatrix}
\begin{bmatrix}
x_1(k)\\
x_2(k)\\
x_3(k) 
\end{bmatrix}$
}
\end{itemize}
  \onslide<6->{
\begin{tiny}The insertion of the $u$'s as a matrix-vector product is left as an exercise. \end{tiny}
}
\onslide<7->{\center --------------------------------------------------------------------}
\end{frame}


% stock model with volume

\begin{frame}
	\frametitle{ARMA Model with Volume Influence on Stock Prices}
	\framesubtitle{Introduction}
	
	\begin{itemize}
		\onslide<2->{
			\item In this presentation, we'll explore an ARMA (AutoRegressive Moving Average) model for stock prices that accounts for the impact of trading volume.
		}
		\onslide<3->{
			\item We'll consider how trading volume can act as a retarding force on the direction of stock price movement.
		}
	\end{itemize}
\end{frame}

\begin{frame}
	\frametitle{ARMA Model Overview}
	\framesubtitle{Basic Concepts}
	
	\begin{itemize}
		\onslide<2->{
			\item ARMA models are used to describe time series data, such as stock prices.
		}
		\onslide<3->{
			\item They consist of two main components: AutoRegressive (AR) and Moving Average (MA) terms.
		}
		\onslide<4->{
			\item The AR component models the stock's dependence on its past values.
		}
		\onslide<5->{
			\item The MA component models the influence of past errors (residuals) on the current value.
		}
	\end{itemize}
\end{frame}


\begin{frame}
	\frametitle{Incorporating Volume}
	\framesubtitle{Volume as a Retarding Force}
	
	\begin{itemize}
		\onslide<2->{
			\item To account for volume's influence, we introduce it as a stochastic variable.
		}
		\onslide<3->{
			\item Trading volume can be used to determine the retarding effect on stock price movements.
		}
		\onslide<4->{
			\item When volume is increasing while the stock is trending down, it may slow or even reverse the trend.
		}
	\end{itemize}
\end{frame}


\begin{frame}
	\frametitle{ARMA Model with Volume}
	\framesubtitle{Equation Setup}
	
	\begin{itemize}
		\onslide<2->{
			\item We extend the traditional ARMA model with an additional term that accounts for trading volume.
		}
		\onslide<3->{
			\item The volume term introduces a retarding effect:
		}
		\onslide<4->{
			\[
			Y_t = \phi_1 Y_{t-1} + \phi_2 Y_{t-2} + \theta_1 \varepsilon_{t-1} + \theta_2 \varepsilon_{t-2} + \eta V_t
			\]
		}
		\onslide<5->{
			Where:
			\begin{itemize}
				\item \(Y_t\) is the stock price at time \(t\).
				\item \(\varepsilon_{t-1}\) and \(\varepsilon_{t-2}\) are lagged error terms.
				\item \(V_t\) is the trading volume at time \(t\).
				\item \(\phi_1, \phi_2, \theta_1, \theta_2, \eta\) are model parameters.
			\end{itemize}
		}
	\end{itemize}
\end{frame}


\begin{frame}
	\frametitle{Volume Impact}
	\framesubtitle{Interpreting Volume Influence}
	
	\begin{itemize}
		\onslide<2->{
			\item When ($\eta$) is positive, an increase in trading volume can have a retarding effect on the stock price.
		}
		\onslide<3->{
			\item This means that a rising stock price may slow down or even reverse when trading volume increases, reflecting possible market sentiment changes.
		}
		\onslide<4->{
			\item The model allows us to capture these dynamics.
		}
	\end{itemize}

\end{frame}
	
	
\begin{frame}
	\frametitle{Conclusion}
	\framesubtitle{ARMA Stock Price Model with Volume}
	
	\begin{itemize}
		\onslide<2->{
			\item The ARMA model with volume influence provides a framework to incorporate trading volume as a retarding force on stock price movements.
		}
		\onslide<3->{
			\item This model can be used to capture complex dynamics in financial markets and enhance forecasting accuracy.
		}
		\onslide<4->{
			\item Further research and data analysis are often necessary to estimate model parameters effectively.
		}
	\end{itemize}
\end{frame}
	


% stock price volume as predator-prey but discrete difference eq.	

\begin{frame}
	\frametitle{Discrete State Vector Model for Stock Prices with Volume Influence}
	\framesubtitle{Introduction}
	
	\begin{itemize}
		\onslide<2->{
			\item In this presentation, we explore a discrete state vector model for stock prices that accounts for the interplay between trading volume and price dynamics.
		}
		\onslide<3->{
			\item The model takes a predator-prey form in discrete difference equations, reflecting the complex relationship between volume and price.
		}
	\end{itemize}
	\onslide<4->{\center --------------------------------------------------------------------}
\end{frame}

\begin{frame}
	\frametitle{Model Overview}
	\framesubtitle{Basic Concepts}
	
	\begin{itemize}
		\onslide<2->{
			\item Our discrete state vector model extends the previous ARMA framework by introducing discrete difference equations for stock price (\(P\)) and trading volume (\(V\)).
		}
		\onslide<3->{
			\item These equations capture the predator-prey relationship, depicting how volume and price influence each other.
		}
		\onslide<4->{
			\item The model reveals the complex interactions between trading activity and stock price movements in discrete time steps.
		}
	\end{itemize}
	\onslide<5->{\center --------------------------------------------------------------------}
\end{frame}


\begin{frame}
	\frametitle{Difference Equations}
	\framesubtitle{Predator-Prey Relationship}
	
	\begin{itemize}
		\onslide<2->{
			\item The discrete state vector model consists of difference equations for stock price (\(P\)) and trading volume (\(V\)).
		}
		\onslide<3->{
			\item The predator-prey relationship in discrete form:
		}
		\onslide<4->{
			\[
			\begin{aligned}
			P_t &= \alpha P_{t-1} - \beta P_{t-1} V_{t-1} \\
			V_t &= \gamma P_{t-1} V_{t-1} - \delta V_{t-1}
			\end{aligned}
			\]
		}
		\onslide<5->{
			Where \(\alpha, \beta, \gamma, \delta\) are model parameters, and \(t\) represents discrete time steps.
		}
	\end{itemize}
	\onslide<6->{\center --------------------------------------------------------------------}
\end{frame}


\begin{frame}
	\frametitle{Interactions}
	\framesubtitle{Volume and Price Dynamics}
	
	\begin{itemize}
		\onslide<2->{
			\item Within this model, trading volume (\(V\)) exerts a retarding effect on the growth of stock price (\(P\)).
		}
		\onslide<3->{
			\item When volume increases while the price is rising, it may slow down or even reverse the trend, mimicking a predator-prey relationship.
		}
		\onslide<4->{
			\item Likewise, stock price influences trading volume, creating a dynamic feedback loop.
		}
	\end{itemize}
	\onslide<5->{\center --------------------------------------------------------------------}
\end{frame}


\begin{frame}
	\frametitle{Complex Dynamics}
	\framesubtitle{Capturing Market Behavior}
	
	\begin{itemize}
		\onslide<2->{
			\item The predator-prey relationship between volume and price captures the intricate dynamics of financial markets in discrete time.
		}
		\onslide<3->{
			\item Market sentiment, trading activity, and price movements are intertwined in this model, revealing complex market behavior.
		}
		\onslide<4->{
			\item This approach enhances our understanding of market dynamics and can improve forecasting accuracy.
		}
	\end{itemize}
	\onslide<5->{\center --------------------------------------------------------------------}
\end{frame}


\begin{frame}
	\frametitle{Conclusion}
	\framesubtitle{}
	
	\begin{itemize}
		\onslide<2->{
			\item The discrete state vector model with a predator-prey relationship between trading volume and stock price offers a powerful tool for analyzing financial market dynamics in discrete time steps.
		}
		\onslide<3->{
			\item This model captures the complex interplay between volume and price, providing insights into real-world financial market behavior.
		}
		\onslide<4->{
			\item It can be valuable for risk assessment, forecasting, and understanding market dynamics in discrete time.
		}
	\end{itemize}
	\onslide<5->{\center --------------------------------------------------------------------}
\end{frame}



	


\begin{frame}
  \onslide<1->{\frametitle{Stability} }
\begin{center}
\includegraphics[width=.9\textwidth,height=.6\textheight]{\dayonesegfour/poincare.jpg}
\end{center}
\begin{tiny}
Source: By Freesodas - Gimp, CC BY-SA 4.0, https://commons.wikimedia.org/w/index.php?curid=67990034
\end{tiny}
\onslide<2->{\center --------------------------------------------------------------------}
\end{frame}

\begin{frame} 
  \onslide<1-> { \frametitle{Software Tools}
    \framesubtitle{MATLAB toolboxes and Simulink blocks}
  }
\begin{itemize}
 \onslide<2-> { 
 \item MATLAB Control System Toolbox
} 
  \begin{itemize}
    \onslide<3-> {
	  \item Linear System Models:  LTI models as either transfer function or state space representation
	}
    \onslide<4-> {
	  \item Functions for interconnecting models in series or parallel
	}
	\onslide<5-> {
	  \item Functions for roundtrip converting from continuous-time to discrete-time
	}
	\onslide<6-> {
      \item Functions for scaling state space models 
	}
  \end{itemize} 
    \onslide<7-> {
	   \item Simulink:  Intuitive Block Diagram GUI Drag and Drop approach (see later slides)
	}
\end{itemize}  
\onslide<8->{\center --------------------------------------------------------------------}
\end{frame}

\begin{frame}
  \onslide<1->{\frametitle{State Space Representation:  Summary} }
\begin{itemize}
  \onslide<2->{
\item State variables sometimes provided by
 $0^\text{th}$ through $n-1^\text{st}$ derivatives of single $n^\text{th}$-order differential equation plus internal dynamic coupling  
}
  \onslide<3->{
\item  Equivalently, outputs of $n$ integrators of system 
}
  \onslide<4->{
\item They represent implicit or explicit internal coupling of the system 
}
  \onslide<5->{
\item In both natural and artificial systems, state space representations are often derived directly from observed or designed couplings 
}
  \onslide<6->{
\item MATLAB Control Systems Toolbox and MATLAB / Simulink have many functions and blocks facilitating state space components and interconnections  
}
  \onslide<7-> {
     \item Stability an important topic but beyond scope of this course
  }
\end{itemize}
\onslide<8->{\center --------------------------------------------------------------------}
\end{frame}

\begin{frame}
	\onslide<1->{\frametitle{Appendix:  State Space Representation}
		\framesubtitle{Matlab / Simulink Example} }
	\begin{itemize}
		\onslide<2->{
			\item Four parameters $A,B,C,D$ from Matlab workspace  
		}
		\onslide<3->{
			\item Measurement model added external to Simulink's state space block  
		}
	\end{itemize}
	\onslide<4->{\center --------------------------------------------------------------------}
\end{frame}

\begin{frame}
	\onslide<1->{\frametitle{Appendix:  State Space Representation}
	\framesubtitle{Matlab / Simulink Example} }
	\begin{center}
		\includegraphics[scale=0.6]{\dayonesegfour/state_space_blocks.jpg}
	\end{center}
	
	\onslide<2->{\center --------------------------------------------------------------------}
\end{frame}

\begin{frame}
	\onslide<1->{\frametitle{Appendix:  State Space Representation}
	\framesubtitle{Matlab / Simulink Example} }
	\begin{center}
		\includegraphics[scale=0.5]{\dayonesegfour/state_space_block_params.jpg}
	\end{center}
	
	\onslide<2->{\center --------------------------------------------------------------------}
\end{frame}

