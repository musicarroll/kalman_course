% day1_seg3_state_space/input_output.tex
\begin{frame}
Input-Output Representation
 \begin{itemize}
 \item Black box approach:  $y(t)=f(u(t),t)$\\

 % day1_seg3_state_space/input_output.tex
\begin{frame}
Input-Output Representation
 \begin{itemize}
 \item Black box approach:  $y(t)=f(u(t),t)$\\

 % day1_seg3_state_space/input_output.tex
\begin{frame}
Input-Output Representation
 \begin{itemize}
 \item Black box approach:  $y(t)=f(u(t),t)$\\

 % day1_seg3_state_space/input_output.tex
\begin{frame}
Input-Output Representation
 \begin{itemize}
 \item Black box approach:  $y(t)=f(u(t),t)$\\

 \input{day1_seg3_state_space/input_output.latex}

 \item Output represents the resultant behavior of the dynamic system as a function of the input (and possibly time).
 \item Output is often one primary scalar variable of interest like position.
\end{itemize}
\end{frame}

\begin{frame}
Input-Output Representation
 \begin{itemize}
 \item Note that output is not necessarily the same thing as the observation or measurement.
 \item Observation or measurement is often related directly to the output through a matrix coupling.
 \item In state space representation, the output is often eliminated in favor of the state vector, and the measurement is given as a function of the state vector directly.
\end{itemize}
\end{frame}

\begin{frame}
Input-Output Representation
 \begin{itemize}
 \item The derivatives of position and how they relate to the forcing function yields a single differential equation.
 \item We'll see later how an $n^\text{th}$-order, scalar differential equation is related to a vector state space model. 
\end{itemize}
\end{frame}


 \item Output represents the resultant behavior of the dynamic system as a function of the input (and possibly time).
 \item Output is often one primary scalar variable of interest like position.
\end{itemize}
\end{frame}

\begin{frame}
Input-Output Representation
 \begin{itemize}
 \item Note that output is not necessarily the same thing as the observation or measurement.
 \item Observation or measurement is often related directly to the output through a matrix coupling.
 \item In state space representation, the output is often eliminated in favor of the state vector, and the measurement is given as a function of the state vector directly.
\end{itemize}
\end{frame}

\begin{frame}
Input-Output Representation
 \begin{itemize}
 \item The derivatives of position and how they relate to the forcing function yields a single differential equation.
 \item We'll see later how an $n^\text{th}$-order, scalar differential equation is related to a vector state space model. 
\end{itemize}
\end{frame}


 \item Output represents the resultant behavior of the dynamic system as a function of the input (and possibly time).
 \item Output is often one primary scalar variable of interest like position.
\end{itemize}
\end{frame}

\begin{frame}
Input-Output Representation
 \begin{itemize}
 \item Note that output is not necessarily the same thing as the observation or measurement.
 \item Observation or measurement is often related directly to the output through a matrix coupling.
 \item In state space representation, the output is often eliminated in favor of the state vector, and the measurement is given as a function of the state vector directly.
\end{itemize}
\end{frame}

\begin{frame}
Input-Output Representation
 \begin{itemize}
 \item The derivatives of position and how they relate to the forcing function yields a single differential equation.
 \item We'll see later how an $n^\text{th}$-order, scalar differential equation is related to a vector state space model. 
\end{itemize}
\end{frame}


 \item Output represents the resultant behavior of the dynamic system as a function of the input (and possibly time).
 \item Output is often one primary scalar variable of interest like position.
\end{itemize}
\end{frame}

\begin{frame}
Input-Output Representation
 \begin{itemize}
 \item Note that output is not necessarily the same thing as the observation or measurement.
 \item Observation or measurement is often related directly to the output through a matrix coupling.
 \item In state space representation, the output is often eliminated in favor of the state vector, and the measurement is given as a function of the state vector directly.
\end{itemize}
\end{frame}

\begin{frame}
Input-Output Representation
 \begin{itemize}
 \item The derivatives of position and how they relate to the forcing function yields a single differential equation.
 \item We'll see later how an $n^\text{th}$-order, scalar differential equation is related to a vector state space model. 
\end{itemize}
\end{frame}
