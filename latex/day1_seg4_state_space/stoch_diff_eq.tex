% day1_seg3_state_space/stoch_diff_eq.tex
\begin{frame}
Stochastic Differential Equations
 \begin{itemize}
  \item If the differential equations describing the state model are driven by white noise, then the resulting solution $x(t)$ would actually have to be a random process.
  \item There is a problem, however:  The white noise model commonly used is not integrable.  Formally, we think of its integral as the Wiener-Levy process.
  \item So, instead of stochastic differential equations driven by white noise, we will briefly examine integral equations driven by the Wiener-Levy process.
  \item Because of this, we need to postpone discussion of stochastic differential equations until after we've discussed probability, statistics, and random process theory.
\end{itemize}
\end{frame}
\begin{frame}
Stochastic Differential Equations
 \begin{itemize}
 \item In this section, we will briefly consider the notion of
 first-order, linear stochastic, vector differential equations of the
 sort:
\begin{equation}\label{stochdiffeq}
  \frac{d}{dt}X(t) = A(t)X(t) + g(t) + WN(t)
\end{equation}
where $X$ and $WN$ are second order vector random processes, $WN$
is in some sense a white noise process, and the initial condition
vector is $X(0)=C$.
 \item However, as we have seen, this $WN$ cannot be the
mean square derivative of the Wiener-Levy vector process, because
the latter is nowhere differentiable in mean square.
\end{itemize}
\end{frame}
\begin{frame}
Stochastic Differential Equations
 \begin{itemize}
 \item Thus we are led to consider the vector integral equation form of
 \eqref{stochdiffeq}:
\begin{equation}\label{inteq}
X(t)=C+\int_0^tA(s)X(s)ds+f(t)+W(t)
\end{equation}
where $W$ is the standard Wiener-Levy vector process.
\end{itemize}
\end{frame}
\begin{frame}
Stochastic Differential Equations
 \begin{itemize}
 \item The homogeneous version of the integral equation
 \eqref{inteq} is
\begin{equation}\label{homointeq}
  X(t)=C+\int_0^tA(s)X(s)ds
\end{equation}
for $t\in [0,T]$.
\item This homogeneous equation has a unique solution $\Phi(t)$,
called the \textit{\textbf{transition matrix}}.
\end{itemize}
\end{frame}
\begin{frame}
Stochastic Differential Equations
 \begin{itemize}
 \item The transition matrix $\Phi(t)$ has the following properties:
 \begin{enumerate}
 \item $X(t)=\Phi(t)X(0)$ for each $t\in [0,T]$. Note that
 $C=X(0)$.
 \item $\frac{d}{dt}\Phi(t)=A(t)\Phi(t)$
 \end{enumerate}
\end{itemize}
\end{frame}
\begin{frame}
Stochastic Differential Equations
 \begin{itemize}
 \item The inhomogeneous mean square solution to equation \eqref{inteq} is
 given by:
\begin{equation}\label{inhomosol}
X(t)=\Phi(t)\left[ C+f(0)+\int_0^t\Phi^{-1}(s)df(s) +
\int_0^t\Phi^{-1}(s)dW(s) \right]
\end{equation}
for each $t\in [0,T]$.
\end{itemize}
\end{frame}
\begin{frame}
 \begin{itemize}
 \item The covariance matrix of a Wiener-Levy vector process is defined as
\begin{equation}\label{covmat}
  \expect{W(s)W\trans (t)}
\end{equation}
and may be represented in  terms of a positive semidefinite
symmetric matrix $B(u)$ as follows:
\begin{equation}\label{covmatrep}
  \expect{W(s)W\trans (t)} = \int_0^s B(u)du
\end{equation}
provided $s\leq t$ on $[0,T]$.
\end{itemize}
\end{frame}
\begin{frame}
Stochastic Differential Equations
 \begin{itemize}
 \item By removing the means and centering the processes, we can
 split the solution \eqref{inhomosol} into a deterministic part and a
 stochastic part.
 \item The deterministic part is
\begin{equation}\label{detpart}
 \expect{X(t)}=\Phi(t)\left[ \expect{C} + f(0) + \int_0^t\Phi^{-1}(s)df(s)
 \right].
\end{equation}
\end{itemize}
\end{frame}
\begin{frame}
Stochastic Differential Equations
 \begin{itemize}
 \item If $X_0(t)=X(t)-\expect{X(t)}$ is the zero-mean stochastic
 process, the stochastic part of the solution to \eqref{inhomosol}
 is given by
\begin{equation}\label{stochpart}
  X_0(t)=\Phi(t)\left[ C_0 + \int_0^t\Phi^{-1}(s)dW(s)\right]
\end{equation}
where $C_0=C-\expect{C}$.
\end{itemize}
\end{frame}
\begin{frame}
Stochastic Differential Equations
 \begin{itemize}
 \item It turns out that the zero-mean state vector covariance has a closed
 form expression.  For $0\leq s\leq t\leq T$ we have:
\begin{equation}\label{centered_statecov}
 \expect{X_0(s)X_0\trans (t)} = \Phi(s)\Psi(s)\Phi\trans (t)
\end{equation}
where
\[\Psi(s)=  \expect{C_0C_0\trans}
 + \int_0^s\Phi^{-1}(u)B(u)[\Phi^{-1}(u)]\trans du\]
 and $B(u)$ is defined by \eqref{covmatrep}.
 \item Note that $X_0(t)$ is thus a zero-mean Gaussian vector
 process.
\end{itemize}
\end{frame}
