\begin{frame}
Ordinary Differential Equations
 \begin{itemize}
 \item With polynomials of differential operators we can now
 define ordinary differential equations.
 \item An equation of the form
 \[p(D)f=q(D)g\]
 where  $f,g\in \Cinf$ and $p(D),q(D)\in \Real(\Diff)$,
 is called an ordinary differential equation with constant
 coefficients.
\end{itemize}
\end{frame}
% orddiffeqs.tex
\begin{frame}
Ordinary Differential Equations
 \begin{itemize}
 \item It is probably more familiar if we use $y$
 instead of $f$ and $u$ instead of $g$, and if we show the
 independent variable, say t, explicitly:
 \[a_n\frac{d^ny}{dt^n}+\cdots +a_1\frac{dy}{dt}+a_0y =
 b_m\frac{d^mu}{dt^m}+\cdots + b_1\frac{du}{dt}+b_0u\]
 or
\begin{equation}\label{diffeq}
\diffeq
\end{equation}
 \item For our purposes, the function $u=u(t)$ is called the \textit{\textbf{input}},
 while the function $y=y(t)$ is called the \textit{\textbf{output}}.  The
 independent variable $t$ is thought of as time.
\end{itemize}
\end{frame}
\begin{frame}
Ordinary Differential Equations
 \begin{itemize}
 \item In \eqref{diffeq}, $u$ is ordinarily assumed to be known or given,
 while $y$ is unknown and sought.
 \item Any function $y$
 satisfying \eqref{diffeq} is called a solution of the
 differential equation.
 \item A differential equation such as \eqref{diffeq} is called
 \textit{\textbf{homogeneous}} when the input function $u$ is identically zero
 for all $t\in[a,b]$.
\end{itemize}
\end{frame}
\begin{frame}
Ordinary Differential Equations
 \begin{itemize}
 \item Thus far we have only dealt with linear differential equations
 with constant coefficients.
 \item We could allow the coefficients to be time-dependent
 functions.  However, for most of our applications, this will not
 be necessary.
\end{itemize}
\end{frame}
% orddiffeqs.tex
\begin{frame}
Ordinary Differential Equations: Examples
 \begin{itemize}
 \item From physics we have Hooke's law: the restoring force of a
 spring is proportional to the distance stretched:  $-ky = m\frac{d^2y}{dt^2}$.
 \item This is just an application of Newton's $F=ma$, where $a$
 is the acceleration: $a=d^2y/dt^2$.
 \item Rewriting this in the form of \eqref{diffeq} we have
 \[m\frac{d^2y}{dt^2}+ky=0\]
 \item Therefore, this is a homogeneous equation of order 2.
\end{itemize}
\end{frame}
% orddiffeqs.tex
\begin{frame}
Ordinary Differential Equations
 \begin{itemize}
 \item We have used $y$ rather than $x$ as in most physics books,
 because we want to be consistent with \eqref{diffeq}.
 \item It is common to rewrite such equations so that the leading
 coefficient is 1.  Thus we have:
 \[\frac{d^2y}{dt^2}+\frac{k}{m}y=0.\]
 \item It is easily verified that $\sin\omega_0 t$, where
 $\omega_0=\sqrt{\frac{k}{m}}$, satisfies this differential equation.
\end{itemize}
\end{frame}
\begin{frame}
Ordinary Differential Equations
 \begin{itemize}
 \item Equation \eqref{diffeq}
 \[\diffeq\]
 is called
 \textit{\textbf{time-invariant}}, because the coefficients are
 constant and also because the independent variable $t$ does not
 appear explicitly in the equation.
 \item Note that \eqref{diffeq} is linear in terms of both the
 input and the output, because the polynomial differential
 operator on each side of the equation is a linear operator.
\end{itemize}
\end{frame}
\begin{frame}
Ordinary Differential Equations
 \begin{itemize}
 \item This double linearity means, on the one hand, that the output $y$ due to several inputs acting
 at the same time is equal to the sum of the outputs due to each
 input acting alone.
 \item On the other hand, it also means that if $y_1$ and $y_2$
 are both solutions to the homogeneous version of \eqref{diffeq},
 then so is $c_1y_1+c_2y_2$ where $c_1$ and $c_2$ are any real
 constants.
 \item A maximal set of $n$ such solutions to the homogeneous
 equation that are also linearly independent (as vectors) is
 called a \textit{\textbf{fundamental set}}.
\end{itemize}
\end{frame}
\begin{frame}
Ordinary Differential Equations
 \begin{itemize}
 \item You can find a fundamental set by solving the
 \textit{\textbf{characteristic equation}} for the equation.  This equation is
 nothing other than the original polynomial equation in $D$, but
 with $D$ replaced by an unknown such as $z$.
 \item Thus, the characteristic equation for \eqref{diffeq} is
 \[a_nz^n+\cdots + a_1z + a_0=0\]
 \item We use $z$ because the roots of this equation are in
 general complex (thanks, once again, to Dr. Gauss).
\end{itemize}
\end{frame}
\begin{frame}
Ordinary Differential Equations
 \begin{itemize}
 \item In solving the characteristic equation, two cases emerge:
 \begin{enumerate}
 \item The roots of the characteristic equation are all distinct.
 \item Some roots are repeated, i.e., each root $r_i$ has
 multiplicity $n_i$.
 \end{enumerate}
\end{itemize}
\end{frame}
\begin{frame}
Ordinary Differential Equations
 \begin{itemize}
 \item In case 1, a fundamental set is easily constructed:
 \[y_1(t)=\exp(r_1t),\ldots,y_n(t)=\exp(r_nt) \]
 where the $r_i$ are the distinct roots.
 \item In case 2, each root $r_i$ of multiplicity $n_i$ contributes
 $n_i$ functions to the fundamental set:
 \[y_{i1}(t)=\exp(r_it),t\exp(r_it),\ldots,t^{n_i-1}\exp(r_it) \]
\end{itemize}
\end{frame}
\begin{frame}
Ordinary Differential Equations
 \begin{itemize}
 \item To fully solve the differential equation, the initial
 conditions need to be taken into account.
 \item These are the values of the function and its first $n-1$
 derivatives at the start of the time interval.
 \item We will specialize our interval $[a,b]$ to the case in which $a=0$
 and $b=T$, some unspecified time in the future.
\end{itemize}
\end{frame}
\begin{frame}
Ordinary Differential Equations
 \begin{itemize}
 \item The \textit{\textbf{free response}} is the solution to the homogeneous
 equation in which the initial conditions are specified but the input is zero, i.e., there is no forcing function.:
\begin{equation}\label{initcond}
  y(0),Dy(0),\ldots,D^{n-1}y(0)
\end{equation}
 \item The \textit{\textbf{forced response}} reverses this situation; it allows the
 input function to be non-zero, but it requires all the initial
 conditions in \eqref{initcond} to be zero.
 \item The \textit{\textbf{total response}} is the sum of the free and forced
 responses.
\end{itemize}
\end{frame}
\begin{frame}
Ordinary Differential Equations: Example of Free, Forced and Total
Responses
 \begin{itemize}
 \item Consider $D^2y+3Dy-1y=u$ with initial conditions:
 $y(0)=1,Dy(0)=-1$.
 \item The roots (found using Matlab's roots function) are $r_1=-3.3028$ and $r_2=0.3028$, and they are
 distinct.
 \item Therefore, a fundamental set is
 $\set{\exp(-3.3028t),\exp(0.3028t)}$.
\end{itemize}
\end{frame}
\begin{frame}
Ordinary Differential Equations: Example of Free, Forced and Total
Responses
 \begin{itemize}
 \item With the fundamental set and the initial conditions we can
 form a set of $n$ linear equations in $n$ unknowns (here, $n=2$):
\begin{align*}
  c_1y_1(0)+c_2y_2(0) & =y(0) \\
  c_1Dy_1(0)+c_2Dy_2(0) & =Dy(0)
\end{align*}
or
\begin{align*}
  1\cdot c_1+1\cdot c_2 & =0 \\
  -3.3028\cdot c_1+0.3028\cdot c_2 & =-1
\end{align*}
\end{itemize}
\end{frame}
\begin{frame}
Ordinary Differential Equations
 \begin{itemize}
 \item The previous linear system is of the form $Ax=b$.  Using
 Matlab's matrix division operator $\backslash$, the solution is
 $x = A\backslash b$ or
\begin{align*}
  c_1 & =    0.2773 \\
  c_2 & =   -0.2773
\end{align*}
%\item We can easily verify that
%$y(t)=0.2773\exp(-3.3028t)-0.2773\exp(0.3028t)$ is indeed the free
%response by plugging and chugging
%\begin{align*}
%  D^2y+3Dy-1y & =D^2[0.2773\exp(-3.3028t)-0.2773\exp(0.3028t)]\\
%   & \quad +3D[0.2773\exp(-3.3028t)-0.2773\exp(0.3028t)]\\
%   & \quad -1[0.2773\exp(-3.3028t)-0.2773\exp(0.3028t)] \\
%   & = 0.2773\cdot 3.3028^2\exp(-3.3028t)\\
%   & \quad -0.2773\cdot 0.3028^2\exp(0.3028t)\\
%   & \quad +3\cdot 0.2773\cdot (-3.3028)\exp(-3.3028t)\\
%   & \quad -3\cdot 0.2773\cdot 0.3028\exp(0.3028t)\\
%   & \quad -0.2773\exp(-3.3028t)+0.2773\exp(0.3028t)
%\end{align*}
\end{itemize}
\end{frame}
\begin{frame}
Ordinary Differential Equations
 \begin{itemize}
 \item The total response can also be decomposed into transient
 and steady state responses
 \item The \textit{\textbf{transient}} response is that part of the total response
 that decays to 0 as $t\rightarrow \infty$.
 \item The \textit{\textbf{steady state response}} is the difference between the
 total response and the transient response.
\end{itemize}
\end{frame}
% day1_seg3_state_space/lti.tex
\begin{frame}
Linear Time-Invariant Differential Equations
 \begin{itemize}
 \item Linear Time-Invariant (LTI) Differential Equations
 \begin{itemize}
 \item Linear: Polynomial differential operators are linear:
\[p(D)[af+bg]= ap(D)f+bp(D)g\]
 \item Time-Invariant means the coefficients are constant.
 \item Homogeneous solution of corresponding state equation is a matrix exponential.  (More on that later when we discuss the state transition matrix.)
\end{itemize}
\end{itemize}
\end{frame}

