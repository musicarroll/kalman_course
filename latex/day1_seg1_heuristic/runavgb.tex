% runavgb.tex
%\begin{comment}
\begin{frame}
\onslide<1->{\frametitle{Recursive Running Average
Python Scripts: dyn\_simple\_moving\_avg\_demo.py}
}
\onslide<2->{
\begin{figure}
\includegraphics[width=0.75\textwidth]{\dayonesegone/runavg_truth_meas_est.png}
\end{figure}
}
\end{frame}

\begin{frame}
  \onslide<1->{
	\frametitle{Running Average} 
	\framesubtitle{Shortcomings}
}
\begin{itemize}
  \onslide<2->{
\item Uses (implicitly) all data points, even those from the "remote" past  
}
  \onslide<3->{
\item Gain gets smaller and smaller ... running average filter goes to sleep, becomes insensitive to new information  
}
  \onslide<4->{
\item Can improve things with a simple moving average (sliding window)  
}
\onslide<5->{\center --------------------------------------------------------------------}
\end{itemize}
\end{frame}

\begin{frame}
  \onslide<1->{
	\frametitle{Simple Moving Average} 
	\framesubtitle{Addressing the Shotcomings}
}
\begin{itemize}
  \onslide<2->{
\item Simple Moving Average of Window Size $M$ only averages the $M$ most recent measurements    
}
  \onslide<3->{
\item Has a fixed gain of $1/M$ which never goes to zero  
}
  \onslide<4->{
\item Assume the number of available data points is $N>M$  
}
  \onslide<5->{
\item  Simple Moving Average of window size $M$ is $s(k,M) = \dfrac{1}{M}\sum_{j=k-M+1}^{k} z(j)$, 
valid for $k\geq M-1$ if indexing starts at 0 (we can set the SMA equal to the running average for $k< M-1$)
}
  \onslide<6->{
\item Recursive SMA formula:  $s(k,M):=s(k-1,M) + \dfrac{1}{M}\left[z(k)-z(k-M+1)\right]$  
}
\onslide<7->{\center --------------------------------------------------------------------}
\end{itemize}
\end{frame}

\begin{frame}
\onslide<1->{\frametitle{Simple Moving Average
Python Scripts: simple\_moving\_avg\_demo.py of window size 10}
}
\onslide<2->{
\begin{figure}
\includegraphics[width=0.75\textwidth]{\dayonesegone/sma_truth_meas_est_w10.png}
\end{figure}
}
\end{frame}


\begin{frame}
  \onslide<1->{
	\frametitle{Running Average} 
	\framesubtitle{Addressing the Shotcomings}
}
\begin{itemize}
  \onslide<2->{
\item  Note that the recursive SMA seems to have shortcomings of its own 
}
  \onslide<3->{
\item   First, it seems overly conservative about updating the previous SMA value
}
\begin{itemize}
  \onslide<4->{ 
    \item However, this is because unchanging process dynamics means that measurements do not change much  
}
  \onslide<5->{ 
    \item Performs worse than running average on constant dynamics, but better with changing dynamics (See next slide)
}
\end{itemize}
  \onslide<6->{
\item   A more aggresive formulation of the recursive SMA can be defined by using a weighting scheme whereby the oldest measurement is 
weighted less than the most recent
}
  \onslide<7->{
\item Neither of these averaging filters explicitly takes into account the degree of uncertainty in the process dynamics   
}
\onslide<8->{\center --------------------------------------------------------------------}
\end{itemize}
\end{frame}

\begin{frame}
\onslide<1->{\frametitle{Simple Moving Average
Python Scripts: dyn\_simple\_moving\_avg\_demo.py of window size 10}
}
\onslide<2->{
\begin{figure}
\includegraphics[width=0.75\textwidth]{\dayonesegone/sma_dyn_truth_meas_est_w10.png}
\end{figure}
}
\end{frame}

\begin{frame}
\onslide<1->{\frametitle{Simple Moving Average
Python Scripts: dyn\_simple\_moving\_avg\_demo.py of window size 5}
}
\onslide<2->{
\begin{figure}
\includegraphics[width=0.75\textwidth]{\dayonesegone/sma_dyn_truth_meas_est_w5.png}
\end{figure}
}
\end{frame}

\begin{frame}
\onslide<1->{\frametitle{Simple Moving Average
Python Scripts: dyn\_simple\_moving\_avg\_demo.py of window size 5}
}
\onslide<2->{
\begin{figure}
\includegraphics[width=0.75\textwidth]{\dayonesegone/runavg_dyn_truth_meas_est.png}
\end{figure}
}
\end{frame}

%\end{comment}
