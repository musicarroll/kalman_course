% predict_correct.tex

\begin{frame}
  \onslide<1->{\frametitle{Recursive Predictor-Corrector Algorithm}
    \framesubtitle{What kinds of problems are we trying to solve?} 
}
\begin{itemize}
  \onslide<2->{\item Problem: To estimate quantities associated with a noisy dynamic process using imperfect, noisy measurements}
\begin{itemize}
  \onslide<3->{\item Convenient to bundle these quantities into a single vector: the \textbf{state vector} of the system or process  }
  \onslide<4->{\item State vector changes with time;  models are not perfect and have noisy inputs (forcing functions) }
  \onslide<5->{\item Can never really know the true state at any given time -- that's why we need an estimator! }
\end{itemize}
  \onslide<6->{\item Measurements have their own challenges}

\begin{itemize}
\onslide<7->{  
	\item Arrive sequentially, possibly event-driven
}
\onslide<8->{  
  \item Out of synch with each other and with the estimation period
}
\onslide<9->{  
  \item Generated by multiple sensors with differing accuracies and noise characteristics
}
\onslide<10->{  
  \item Don't necessarily measure the states directly (need a measurement model to relate measurements to the states)
}
\end{itemize}

\end{itemize}
\onslide<11->{\center --------------------------------------------------------------------}
\end{frame}

\begin{frame}
  \onslide<1->{\frametitle{Recursive Predictor-Corrector Algorithm}}
\begin{itemize}
  \onslide<2->{\item Goal:  Devise a method to combine all the measurement information and produce 
the best estimate of the state of system}
\begin{itemize}
  \onslide<3->{\item Generate estimates in real-time, if possible, and avoid growing memory consumption}
\end{itemize}
  \onslide<4->{\item Seek an algorithm to:}
\begin{itemize}
  \onslide<5->{\item Predict how the system is changing over time between measurements}
  \onslide<6->{\item Correct our predicted state estimates with new measurement}
  \onslide<7->{\item Avoid growing memory requirements}
\end{itemize}
  \onslide<7->{\item Bottom Line:  We seek a \textbf{recursive} predictor-corrector algorithm}
\end{itemize}
\onslide<8->{\center --------------------------------------------------------------------}
\end{frame}

\begin{frame}
\onslide<1->{\frametitle{
Recursive Predictor-Corrector Algorithm
}}
\begin{itemize}
 
  \onslide<2->{\item Prediction:
 }
\begin{itemize}
 
  \onslide<3->{\item \textit{\textbf{Extrapolate}} solution from  previous iteration to present time
}
  \onslide<4->{(This is how we take system dynamics into account)} 
\end{itemize}

\onslide<5->{\item Correction:
 }
\begin{itemize}
  \onslide<6->{\item Use measurement information to 
     \textit{\textbf{update}} or \textit{\textbf{correct}} the prediction}
 
\end{itemize}
 
  \onslide<7->{\item And the cycle recursively begins anew!
}
\end{itemize}
\onslide<8->{\center --------------------------------------------------------------------}
\end{frame}

\begin{frame} 
\onslide<1->{\frametitle{Kalman Filter Overview}}
\framesubtitle{Some Notational Conventions}
\begin{itemize}
\onslide<2->{\item Time-dependent scalar or vector quantities:\\
  $x(k)$ (discrete), $x(t)$ (continuous) }
\onslide<3->{\item Corresponding vector components:\\
$x_i(k)$ or $x_i(t)$ }
\onslide<4->{\item Time-dependent matrix quantities (capital letters):\\
  $P(k)$ (discrete), $P(t)$ (continuous) }
\onslide<5->{\item Corresponding matrix components (elements):\\
 $p_{ij}(k)$ or $p_{ij}(t)$ }
\onslide<6->{
  \item Use hats for estimated quantities and special superscripts to indicate pre- and post-update values:  \\
$\hat{x}^{+}(k)$ and $P^{-}(k)$
}
\onslide<7->{
  \item Use tildes for error quantities, i.e., the difference between the estimate and the truth:  \\
$\tilde{x}(k):=\hat{x}(k)-x(k)$
}
\end{itemize}
\onslide<8->{\center --------------------------------------------------------------------}
\end{frame}


\begin{frame}

\frametitle{Recursive Predictor-Corrector Algorithm (Discrete System)
}
\begin{figure}
   \includegraphics[width=0.75\textwidth]{\dayonesegone/pred_correct.jpg}
\end{figure}
\onslide<1->{\center --------------------------------------------------------------------}

\end{frame}


\begin{frame}
  \onslide<1->{
\frametitle{Recursive Predictor-Corrector Algorithm
} }
\begin{itemize}
 
  \onslide<2->{\item Algorithm
  }
\begin{itemize}
 
  \onslide<3->{\item Computational procedure
 }
  \onslide<4->{\item Given initial values of some quantities, compute new quantities (using system dynamics and measurement info)
 }
\end{itemize}
  \onslide<5->{\item Recursive
 }
\begin{itemize}
  \onslide<6->{\item New quantities are same type as the previous
 }
  \onslide<7->{\item New quantities become initial values for next cycle
 }
\end{itemize}

\end{itemize}

\onslide<8->{\center --------------------------------------------------------------------}
\end{frame}
