% \dayonesegone/exercises.tex

\begin{enumerate}
\begin{frame}

  \frametitle{Exercises}
\item Let's re-familiarize ourselves with vectors and matrices.  
Addition:
\[\colvec{x}{n}+\colvec{y}{n}=\colvecsum{x}{y}{n}\]
For vectors in three space, this would be:

\[\colvecthree{x}{}+\colvecthree{y}{}=\colvecthreesum{x}{y}\] 

Compute:
\[
\begin{bmatrix}
-1\\
\;\;\;3\\
\;\;\;0 \end{bmatrix}
+
\begin{bmatrix}
\;\;\;7\\
-1\\
\;\;\;5 \end{bmatrix}=
\]

\end{frame}


\begin{frame}

  \frametitle{Exercises}
\[
\begin{bmatrix}
0\\
0\\
0 \end{bmatrix}
+
\begin{bmatrix}
2\\
2\\
1 \end{bmatrix}=
\]

\[
\begin{bmatrix}
0.75\\
\;\;-3\\
100 \end{bmatrix}
+
\begin{bmatrix}
\;\;\;0.75\\
\;\;\;\;\;\;0\\
-1000 \end{bmatrix}=
\]

\end{frame}


\begin{frame}

  \frametitle{Exercises}
\[
\begin{bmatrix}
1\\
0\\
0 \end{bmatrix}
+
\begin{bmatrix}
0\\
1\\
0 \end{bmatrix}=
\]

\[
\begin{bmatrix}
\;\;\;1.1\\
-3.2\\
\;\;\;9.7 \end{bmatrix}
+
\begin{bmatrix}
-0.8\\
\;\;\;3.2\\
\;\;\;10 \end{bmatrix}=
\]

\end{frame}


\begin{frame}
 
  \frametitle{Exercises}
\item Multiplying a row vector times a column vector:

\[
\begin{bmatrix}
a_1&a_2&a_3
\end{bmatrix}
\begin{bmatrix}
b_1\\
b_2\\
b_3
\end{bmatrix}
=
a_1b_1 + a_2b_2 + a_3b_3
\]

Now you try it:
\[
\begin{bmatrix}
0&1&3
\end{bmatrix}
\begin{bmatrix}
4\\
3\\
0
\end{bmatrix}
=
\]

\[
\begin{bmatrix}
-1&3&-2
\end{bmatrix}
\begin{bmatrix}
\;\;1\\
-2\\
\;\;2
\end{bmatrix}
=
\]

\end{frame}


\begin{frame}

  \frametitle{Exercises}
\[
\begin{bmatrix}
x&y&z
\end{bmatrix}
\begin{bmatrix}
u\\
v\\
w
\end{bmatrix}
=
\]

\[
\begin{bmatrix}
2x&.5x&-7x
\end{bmatrix}
\begin{bmatrix}
-10\\
\;\;\;4\\
\;-1
\end{bmatrix}
=
\]

\end{frame}


\begin{frame}


  \frametitle{Exercises}
\item Multiplying a matrix times a column vector:

\[
\begin{bmatrix}
1&0&1\\
2&7&0\\
3&2&5
\end{bmatrix}
\begin{bmatrix}
\;\;\;7\\
-1\\
\;\;\;5 
\end{bmatrix}=
\]

Hint: Multiply the first row of the matrix times the column vector to get the first element of the new column vector.  
Then multiply by the second row of the matrix, and so on.
\[
\begin{bmatrix}
2&1&0\\
0&1&1\\
2&2&2
\end{bmatrix}
\begin{bmatrix}
\;\;\;3\\
-3\\
\;\;\;1 
\end{bmatrix}=
\]

\[
\begin{bmatrix}
\;\;3&1&\;\;1\\
-1&1&-1\\
\;\;0&2&\;\;2
\end{bmatrix}
\begin{bmatrix}
\;\;\;3\\
-3\\
\;\;\;1 
\end{bmatrix}=
\]

\end{frame}


\begin{frame}

  \frametitle{Exercises}
\item What happens if you multiply a column on the left by a row vector on the right?\\


item What is the transpose of the following matrix?
\[
\begin{bmatrix}
a_{11}&a_{12}&a_{13}\\
a_{21}&a_{22}&a_{23}\\
a_{31}&a_{32}&a_{33}\\
\end{bmatrix}
\]

\end{frame}


\begin{frame}

  \frametitle{Exercises}
\item Given the following scalar measurements of some constant scalar process, calculate the average value at each stage $k$ by two different methods: 1) by averaging all past values and 2) by using the recursive algorithm in Eq. $\eqref{recursiverunavg}$:
	\begin{enumerate}
		\item $z_1=0.76$
		\item $z_2=0.67$
		\item $z_3=0.79$
		\item $z_4=0.81$
		\item $z_5=0.78$
		\item $z_6=0.79$
		\item $z_7=0.81$
		\item $z_8=0.82$
		\item $z_9=0.77$
	\end{enumerate}

\end{frame}


\begin{frame}
 
  \frametitle{Exercises}
\item You are traveling in a car at 45 mph in a 35 mph speed
 zone.  Your friendly neighborhood police officer is watching you with
 her radar gun.  
She is measuring your speed, of course, but needs
 to report as well your position as a function of time.  Write down 
 a dynamics and 
measurement model for this problem in which the state variable is your
 positional displacement from some arbitrary initial position.  
Assume that
 there is no process noise in your dynamics and that your velocity
 is not changing.  Assume the policeman makes a measurement 
with
 her radar gun every 2 seconds.\\
Hint: You have two variables of interest!

\item How is this problem different from the resistor problem?

\end{frame}
\end{enumerate}