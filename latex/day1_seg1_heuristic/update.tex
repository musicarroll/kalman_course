% \dayonesegone/update.tex
\begin{frame}
  \onslide<1->{
	\frametitle{Correction (Measurement Update)}
	\framesubtitle{The Measurement Model}
	}
 \begin{itemize}
  \onslide<2->{\item In the running average example, the update implicitly
 assumed that measurements were corrupted by white noise}
   \onslide<3->{\item Measurement model for this would be:\\ $z(k)=x(k)+v(k)$, \\where $x(k)$ is the true state and $v(k)$ is
   measurement noise sequence}
   \onslide<4->{\item Simulated measurements are based on this:  We provide a truth model and then add normally distributed 
   measurement noise samples at time step $k$}
   \onslide<5->{\item Measurement processing is sometimes called \bfit{Observation} processing}
\end{itemize}
\onslide<6->{\center --------------------------------------------------------------------}
\end{frame}

\begin{frame}
  \onslide<1->{\frametitle{Correction (Measurement Update)}}
 \begin{itemize}
   \onslide<2->{\item Because we assumed that measurement noise was white
 Gaussian and zero mean, our best guess was simply the average value
 of all the measurements}
   \onslide<3->{\item Recall the resistor example:  Voltmeter not
 perfect and each measurement differs slightly from previous
 one due to measurement noise, i.e., randomness in the voltmeter}
   \onslide<4->{
	\item This randomness is completely independent of any noise that might be disturbing the dynamic process itself
   }
   
\end{itemize}
\onslide<5->{\center --------------------------------------------------------------------}
\end{frame}

\begin{frame}
  \onslide<1->{\frametitle{Correction (Measurement Update)}}
 \begin{itemize}
 \onslide<2->{\item General Kalman filtering complicates the measurement model
 by adding two new elements:}
 \begin{itemize}
 \onslide<3->{\item The measurement may have a more complicated relationship to
 the underlying system state:  
\[z(k)=h\left[k,x(k)\right]+v(k),\] 
where $h$ could be a non-linear function with possible explicit time dependence}
 \onslide<4->{\item Just as the state $x$ is allowed to be a vector variable, so
 the measurement $z$ could also be a vector variable}
 \onslide<5->{\item This means that we can measure several different quantities simultaneously, or use redundant measurements of the same 
 quantities from different sensors
 }
 \onslide<6->{
	\item Note that the measurement model is a direct algebraic equation, not a differential equation nor a difference equation
 }
 
 \end{itemize}
\end{itemize}
\onslide<6->{\center --------------------------------------------------------------------}
\end{frame}

\begin{frame}
\onslide<1->{
	\frametitle{Correction (Measurement Update)}
	\framesubtitle{Recursive Running Average Revisited}
}
 \begin{itemize}
 \onslide<2->{
	\item Recall the recursive formulation of the running average
 }
 \onslide<3->{
	\item We used the latest measurement to \textbf{update} (or \textbf{correct}) the last estimate:
}
\onslide<4->{\begin{equation}\label{recursiverunavgupdate}
\stateest{k}{+})=\stateest{k}{-}+K\left[z(k)-\stateest{k}{-}\right]
\end{equation} 
\;\;\;\;\;\;\;where the gain is $K=\dfrac{1}{n}$
}
 \onslide<5->{
	\item Note the use of $\stateest{k}{-}$ in the measurement residual.  Ordinarily, we would use $h(\stateest{k}{-})$ instead because the measurement is a function of the state
	(e.g., it might be in different units of measurement than the state variable)
}

\end{itemize}

\onslide<6->{\center --------------------------------------------------------------------}
\end{frame}

\begin{frame}
  \onslide<1->{
	\frametitle{Correction (Measurement Update)} 
	\framesubtitle{Simultaneous Measurements}
}
\begin{itemize}
  \onslide<2->{
\item  If we had two voltmeters and make simultaneous measurements of the resistance, our measurement model would look like this 
}
  \onslide<3->{
   \[\colvectwo{z}{2}{k}=
   \begin{bmatrix}
   1 & 0 \\
   0 & 1
   \end{bmatrix}
   \colvectwo{x}{2}{k}+\colvectwo{v}{2}{k}\]  
}
  \onslide<4->{
\item Here the $h$ function is a simple 2x2 identity matrix
}
  \onslide<5->{
\item  The noise for each voltmeter would likely be different, unless these units were considered to be identical in precision
}
  \onslide<6->{
\item  The solution of the Kalman gain assumes that the $h$ is a matrix; the letter $H$ is conventionally reserved for the measurement matrix 
}
\end{itemize}
\onslide<7->{\center --------------------------------------------------------------------}
\end{frame}
