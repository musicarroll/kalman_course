% \dayonesegone/heuristic_main.tex
\begin{frame}
  \onslide<1->{\frametitle{Part I}
  \framesubtitle{The Basic Kalman Equations\\
  Topics}
  }
 \begin{itemize}
  \onslide<2->{\item Understanding the Equations: Heuristic Introduction}
  \onslide<3->{\item Equation Drilldown:  Taking the Equations Apart}
%  \onslide<4->{\item Differential Equations, Difference Equations and Dynamic Systems}
  \onslide<4->{\item State Space Concepts}
\end{itemize}
\onslide<5->{\center --------------------------------------------------------------------}
\end{frame}

\begin{frame}
\onslide<1->{
	\frametitle{The Basic Kalman Equations}
	\framesubtitle{Understanding the Equations: Heuristic Introduction\\
					Subtopics}
}
\begin{itemize}
 \onslide<2->{\item Recursive Predictor-Corrector Algorithms}
 \onslide<3->{\item Running Averages}
 \onslide<4->{\item Prediction (Extrapolation)}
 \onslide<5->{\item Correction (Measurement Update)}
 \onslide<6->{\item \textbf{Gain Computation}}
\end{itemize}
\onslide<7->{\center --------------------------------------------------------------------}
\end{frame}

% \dayonesegone/gain.tex
\begin{frame}
  \onslide<1->{\frametitle{Kalman Gain Computation}}
 \begin{itemize}
  \onslide<2->{
		\item Technically, the \bfit{Kalman Gain} is computed \bfit{before} the measurement update is performed}
   \onslide<3->{\item To blend information from \textbf{previous estimate} and \textbf{current measurement}, we need to  
   calculate the Kalman gain, denoted by $K(k)$}
  \onslide<4->{ \item Recall running average example: $K(k)=\dfrac{1}{k}$, a simple scalar}
  \begin{itemize}
	\onslide<5->{ \item If we have $k$ pieces of information \textit{of equal quality}, then, 
			intuitively, it makes sense weight each piece equally using a factor of $1/k$}
	\end{itemize}
\end{itemize}
\onslide<6->{\center --------------------------------------------------------------------}
\end{frame}

\begin{frame}
  \onslide<1->{\frametitle{Kalman Gain Computation}}
 \begin{itemize}
 \onslide<2->{\item Quantity to be weighted is the \textbf{measurement residual}:  
	\[z(k)-H\stateest{k}{-}\]
	i.e., the difference between the \textbf{actual} and \textbf{predicted} measurements}
 \onslide<3->{\item Residual contains all new information, i.e., new information from most recent 
 measurement minus what we think we already knew, i.e., the prediction based on previous
 estimate}
 \onslide<4->{\item Gain computation occurs after the prediction step but before the update step; sometimes thought of a preliminary sub-step within the update step}
 \onslide<5->{\item Sometimes referred to as the 'innovation gain' (since the residual is somethimes called the 'innovation'}
\end{itemize}
\onslide<6->{\center --------------------------------------------------------------------}
\end{frame}

\begin{frame}
  \onslide<1->{
	\frametitle{Kalman Gain Computation} 
	\framesubtitle{Intuitive Understanding of the Gain}
}
\begin{itemize}
  \onslide<2->{
\item Kalman gain computation is the heart of the Kalman algorithm  
}
  \begin{itemize}
	\onslide<3->{
		\item Guarantees optimality  
	}
  \end{itemize}
  \onslide<4->{
\item  Intuitively, gain is a function of two kinds of uncertainty:   
}
\begin{itemize}
  \onslide<5->{
	\item Uncertainty in previous estimate:  How precise was our last estimate?   
}
\begin{itemize}
  \onslide<6->{
	\item  Dependent on uncertainty in the process model:  
		How well have we actually modeled the system dynamics and how much do uncontrollable disturbances influence our estimates? 
}
\end{itemize}

  \onslide<7->{
	\item Uncertainty in the measurement:  How precise are the measurements?   
}
\end{itemize}
  \onslide<8->{
\item  Gain weights the innovation, and these two kinds of uncertainty are all implicity present in the innovation
}
\end{itemize}
\onslide<9->{\center --------------------------------------------------------------------}
\end{frame}

\begin{frame}

\frametitle{Recursive Predictor-Corrector Algorithm}
\framesubtitle{Discrete System}
\begin{figure}
   \includegraphics[width=0.75\textwidth]{\dayonesegone/pred_correct.jpg}
\end{figure}
\onslide<1->{\center --------------------------------------------------------------------}
\end{frame}

\begin{frame}
\frametitle{Kalman Gain}
\framesubtitle{within Correction (or Update) Block}

\begin{figure}
   \includegraphics[width=0.75\textwidth]{\dayonesegone/correction_block.jpg}
\end{figure}
\onslide<1->{\center --------------------------------------------------------------------}

\end{frame}




\begin{frame}
  \onslide<1->{
	\frametitle{Gain Summary} 
	\framesubtitle{Vector Check}
}
\begin{itemize}
  \onslide<2->{
\item Where are we?  
}
\begin{itemize}
  \onslide<3->{
\item Examined the concept of Kalman gain in more detail  
}
  \onslide<4->{
\item Considered gain as weight for measurement residual / innovation 
}
\end{itemize}
  \onslide<5->{
\item What's next?  
}
\begin{itemize}
  \onslide<6->{
\item In the next playlist we will present in more detail \bfit{the Five Kalman equations}   
}
  \onslide<7->{
\item We will analyze what the five equations mean   
}
\end{itemize}
\end{itemize}
\onslide<8->{\center --------------------------------------------------------------------}
\end{frame}


