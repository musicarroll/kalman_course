% \dayonesegone/runavg.tex
\begin{frame}
  \onslide<1->{\frametitle{Simple Example:  Running Averages}}
 \begin{itemize}
  \onslide<2->{ \item Suppose the boss says that we need to estimate the resistance of a particular resistor.  We are given ...}
	\begin{itemize}
		\onslide<3->{ \item The manufacturer's estimate of the true value, which has some uncertainty associated with it}
		\onslide<4->{ \item A dynamic model of how the resistance: The (unknown) "true" value $x$ of the resistance is assumed to be
			constant in time: $x(k) = x(k-1)$}
		\onslide<5->{\item A voltmeter with which to make measurements of the resistance in Ohms}
	\end{itemize}
  \onslide<6->{ \item How should we go about coming up with a better estimate of the resistance?}
  \onslide<7->{ \item Our plan:  Take repeated measurements using the voltmeter to obtain a sequence of measurement values:}
  \begin{itemize}
	\onslide<8->{ \item $z(1), z(2), \ldots$}
	\onslide<9->{ \item ... and then average all the values}
  \end{itemize}
  
\end{itemize}
\onslide<10->{\center --------------------------------------------------------------------}
\end{frame}

\begin{frame}
\onslide<1->{\frametitle{Recursive Running Average}}
\onslide<2->{\begin{figure}

\includegraphics[width=0.75\textwidth]{\dayonesegone/runavg_truth_meas.png}
\end{figure}}
\onslide<3->{\center --------------------------------------------------------------------}
\end{frame}

\begin{frame}
  \onslide<1->{\frametitle{Simple Example: Running Averages}}
  \begin{itemize}
	\onslide<2->{\item The \textbf{\textit{state}} that we are trying to estimate is the true resistance value of the resistor}
	\onslide<3->{\item Our \textbf{\textit{state dynamics}}  or \textbf{\textit{process model}} is very simple:  
		\[x(k) = x(k-1)\]}
  \onslide<4->{ \item The state is a scalar quantity (1-dimensional) and does not change with time}
\end{itemize} 
\onslide<5->{\center --------------------------------------------------------------------}
\end{frame}


\begin{frame}
  \onslide<1->{\frametitle{Simple Example: Running Averages}}
\begin{itemize}
  \onslide<2->{\item What is the best way to estimate the state of our system?}
  \onslide<3->{\item In the absence of any other information, we should probably just average the measurements}
  \begin{itemize}
	\onslide<4-> {
		\item We have no \textit{a priori} reason to believe that any one measurement is better than any other and we trust our voltmeter to be unbiased
	}
	\onslide<5-> {
		\item Therefore, we ought to weight them equally compute a simple average (sample mean of measurements)
	}
  \end{itemize}
  \onslide<6->{
	\item We assume such an average is unbiased, because the measurement errors ought average out to zero (zero-mean)
  }
\end{itemize} 
\onslide<7->{\center --------------------------------------------------------------------}
\end{frame}

\begin{frame}
  \onslide<1->{\frametitle{Simple Example:  Running Averages}}
 \begin{itemize}
  \onslide<2->{ 
	\item A simple running average is very much like a Kalman filter with
		simple, unchanging dynamics ($\;\;x(k)=x(k-1)\;\;$)
  }
  \onslide<3->{
	\item Given a sequence of (noisy) measurements, denoted $z(1),z(2),\ldots$, compute at each step (epoch) 
		the best estimate $\stateest{k}{+}$ of the true value of the state:  \\At the $k^{\text{th}}$
		step, the best guess would be the average of all acquired measurement values:
  }
\end{itemize}
  \onslide<4->{
	\begin{equation}\label{runavg}
		\stateest{k}{+}=\dfrac{1}{k}\sum_{j=1}^{k}z(j)=\dfrac{1}{k}[\;z(1)+z(2)+\cdots+z(k)\;]
	\end{equation}
  }
\onslide<5->{\center --------------------------------------------------------------------}
\end{frame}

\begin{frame}
 \onslide<1->{\frametitle{Simple Example:  Running Averages}}
 \begin{itemize}
 \onslide<2->{\item Q: Why is the average of all measurements the "best" estimate?}
 \onslide<3->{\item A: The dynamics tell us that the quantity to be estimated is not changing, and therefore 
the only variability stems from the measurements, i.e., the measurement noise}
 \onslide<4->{\item Thus, we should try to eliminate the measurement variability by averaging the measurements}
 \onslide<5->{\item Note that in this example $\stateest{k}{-}=\stateest{k-1}{+}$ due to the constant dynamics;  
 $\stateest{k}{+}$ adds the new measurement and averages over all $k$ measurements }
\end{itemize}
\onslide<6->{\center --------------------------------------------------------------------}
\end{frame}

\begin{frame}
\onslide<1->{\frametitle{Recursive Running Average}}
\begin{itemize}
 \onslide<2->{ \item Therefore, as much as possible, we want to rid ourselves of the noise by averaging it out}
 \onslide<3->{ \item We assume that our voltmeter is good, i.e., that the measurement noise is zero-mean, i.e., 
   its average tends to zero over the long run}
 \onslide<4->{ \item If we had a bias in the voltmeter, we wouldn't call that noise; that would be a systematic error on the 
 part of our voltmeter.  We'd have to calibrate it out.}
  \onslide<5->{\item But noise can't be calibrated out; it must be averaged or \textbf{filtered} out over time}
\end{itemize}
\onslide<6->{\center --------------------------------------------------------------------}
\end{frame}

\begin{frame}
\onslide<1->{\frametitle{Recursive Running Average}}
\onslide<2->{\begin{figure}

\includegraphics[width=0.75\textwidth]{\dayonesegone/runavg_truth_meas_est.png}
\end{figure}}
\onslide<3->{\center --------------------------------------------------------------------}
\end{frame}


\begin{frame}
\onslide<1->{\frametitle{Recursive Running Average}}
\onslide<2->{Eq.\eqref{runavg}\\
\[\stateest{k}{+}=\dfrac{1}{k}\sum_{j=1}^{k}z(j)=\dfrac{1}{k}[\;z(1)+z(2)+\cdots+z(k)\;]\]}
\begin{itemize}
\onslide<3->{ \item Now suppose our estimating task is going to continue for a long time and we don't want to fill up our
computer memory with a record of all the measurements taken since the start}
\onslide<4->{ \item We just want to use the previously calculated value and add a small increment}
\end{itemize}
\onslide<5->{\center --------------------------------------------------------------------}
\end{frame}

\begin{frame}
  \onslide<1->{
	\frametitle{Recursive Running Average} 
	\framesubtitle{Derivation}
}
\begin{itemize}
  \onslide<2->{
\item What's the difference between $\stateest{k}{+}$ and $\stateest{k-1}{+}$\\
(which is the same as $\stateest{k}{-}$ in this case)?
}
  \onslide<3->{
\item   Eq.\eqref{runavg} tells us that:  $\sum_{j=1}^{k-1}z(j)=(k-1)\stateest{k-1}{+}$
}
  \onslide<4->{
\item   We now compute the difference as
\[k\,\stateest{k}{+}-(k-1)\stateest{k-1}{+}=z(k)\] 
and, solving for $\stateest{k}{+}$, we get
\[\stateest{k}{+}=\dfrac{k-1}{k}\stateest{k-1}{+}\dfrac{1}{k}+z(k)=\dfrac{k-1}{k}\stateest{k}{-}+\dfrac{1}{k}z(k)\]
}
  \onslide<5->{
\item Note that the last expression on the right is a weighted average of two terms:  The old information and the new.Now we can rearrange the right side and obtain our recursive formula for the running average ...   
}
  \onslide<6->{
\item Now we can further rearrange the right side and obtain our final recursive formula for the running average ...   
}
\onslide<7->{\center --------------------------------------------------------------------}

\end{itemize}
\end{frame}


\begin{frame}
\onslide<1->{Step-by-Step Derivation:}
\begin{footnotesize} 
\begin{align*}
  \onslide<2->{\stateest{k}{+} &=\dfrac{1}{k}\sum_{j=1}^{k}z(j) \text{\;\;\;\;\;\;\;\;\;\;\;\;\;\;\;\;\;\;\;\;\;\;\;\;\;\;\;\;\;\;\;\;\;\;\;by def. Eq. }\eqref{runavg}\\}
  \onslide<3->{&=\dfrac{1}{k}\sum_{j=1}^{k-1}z(j)+\dfrac{1}{k}z(k)  \text{\;\;\;\;\;\;\;\;\;\;\;\;\;\;\;\;\;\;\;\; separate last term from sum} \\}
  \onslide<4->{&=\dfrac{1}{k}(k-1)\stateest{k}{-}+\dfrac{1}{k}z(k) \text{\;\;\;\;\;\;\;\;\;\;\;\; replace the sum using Eq. } \eqref{runavg}\text{ again} \\ }
  \onslide<5->{&=\dfrac{k}{k}\stateest{k}{-} -\dfrac{1}{k}\stateest{k}{-} +\dfrac{1}{k}z(k)\text{\;\;\;\;\;\;\; distribute }\\}
  \onslide<6->{&=\stateest{k}{-}+\dfrac{1}{k}\left[\;z(k)-\stateest{k}{-}\;\right].\text{\;\;\;\;\;\; simplify, rearrange and factor out 1/$k$ }}
\end{align*}
\end{footnotesize}
\onslide<7->{\center --------------------------------------------------------------------}
\end{frame}

\begin{frame}
\onslide<1->{\frametitle{\textit{\textbf{Recursive}} Running Average}}
  \begin{itemize}
   \onslide<2->{\item We now have a recursive algorithm for computing a running average. \\\;\;Procedure:}
	 \begin{itemize}
		\onslide<3->{\item At the $k^{\textit{th}}$ step take the current measurrement and 
			subtract off from it the old average to obtain the residual }
		\onslide<4->{\item Multiply the residual by a gain factor of $1/k$ }
		\onslide<5->{\item Finally, add this weighted residual to the old average to obtain the new average}
	 \end{itemize}
  \end{itemize}
\onslide<6->{\center --------------------------------------------------------------------}
\end{frame}

\begin{frame}
\onslide<1->{\frametitle{Recursive Running Average}}
 \begin{itemize}
 \onslide<2->{\item Note that we don't have to remember all the previous
 values or averages, only the last average; we use the latest measurement to \textbf{correct} the last average
 }
 \onslide<3->{ \item Note also that as $k$ goes to $\infty$, the gain $1/k$ goes
 to 0. \\(The filter is said to be going to sleep!)}
\end{itemize}
\onslide<4->{
\begin{equation}\label{recursiverunavg}
  \stateest{k}{+}=\stateest{k}{-}+\dfrac{1}{k}\left[z(k)-\stateest{k}{-}\right]
\end{equation} 
}
\onslide<5->{\center --------------------------------------------------------------------}
\end{frame}

\begin{frame}
\onslide<1->{\frametitle{Recursive Running Average}
	\framesubtitle{The Measurement Residual}
	}
 \begin{itemize}
  \onslide<2->{\item The quantity in brackets, $z(k)-\stateest{k}{-}$, is called the 
  \textbf{\textit{measurement residual}}, \textbf{\textit{pre-fit residual}} or the 
  \textbf{\textit{innovation}}}
  \onslide<3->{\item It is the difference between the actual new measurement and the predicted new measurement}
  \onslide<4->{\item Q: Why is $\stateest{k}{-}$ our predicted measurement?}
  \onslide<5->{\item A: We have assumed that the state in this case is not changing with time.  Therefore, in the absence of new 
  information the previous estimate is the best answer thus far!}
  \begin{itemize}
    \onslide<6->{\item When non-constant state dynamics come into play, the predicted measurement will certainly differ from 
	  the previous updated estimate, but we expect this predicted measurement to be fairly close to the actual measurement \\
	  (if our dynamic model is any good)}
  \end{itemize}
 \end{itemize}
\onslide<7->{\center --------------------------------------------------------------------}
\end{frame}


