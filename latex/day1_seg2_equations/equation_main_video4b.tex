% \dayonesegtwo/equation_main.tex
\begin{frame}
  \onslide<1->{\frametitle{Part I}
  \framesubtitle{The Five Basic Kalman Equations\\
  Topics}
  }
 \begin{itemize}
  \onslide<2->{\item Understanding the Equations: Heuristic Introduction}
  \onslide<3->{\item \textbf{Equation Drilldown:  Taking the Equations Apart}}
%  \onslide<4->{\item Differential Equations, Difference Equations and Dynamic Systems}
  \onslide<4->{\item State Space Concepts}
\end{itemize}
\onslide<5->{\center --------------------------------------------------------------------}
\end{frame}


\begin{frame}

\onslide<1->{
  \frametitle{Equation Drilldown:  Taking the Equations Apart}
     \framesubtitle{Topics}
}

\begin{itemize}
  \onslide<2->{\item Mathematical Formulation of the Problem}
  \onslide<3->{\item Drilldown on State Dynamics and Covariance Extrapolation Equations}
  \onslide<4->{\item The Five Kalman Filter Equations}
  \onslide<5->{\item \textbf{Examples:  Constant Velocity Motion}}
  \onslide<6->{\item Exercises}
\end{itemize}

\onslide<7->{\center --------------------------------------------------------------------}
\end{frame}

% Example:  Constant Velocity (non-accelerated motion)

\begin{frame}
  \onslide<1->{\frametitle{Non-Accelerated Motion } 
  \framesubtitle{Constant Velocity}
  }
\begin{itemize}
  \onslide<2->{\item Number of states: 2  }
  \onslide<3->{\item System Dynamics:
      \begin{align*}
	x_1(k)&=x_1(k-1)+x_2(k-1)\Delta t\\
	x_2(k)&=x_2(k-1)
      \end{align*} }
  \onslide<4->{\item Measurement Model:  $z(k)=x_1(k)+v(k)$, where $v(k)$ is zero mean, 
Gaussian white noise with variance $\sigma^2$}
    \onslide<5->{\item Thus, $\Phi  = \begin{bmatrix}
1 & \Delta t\\
0 & 1
\end{bmatrix}$, $\Delta t=t_k-t_{k-1}$ (assumed constant), $H_k=\begin{bmatrix}1 & 0\end{bmatrix}$, $Q_k=0$, $R_k=\sigma^2$}
  \onslide<6->{\item Note that we are assuming the acceleration is truly zero.  Any uncertainty 
in that assumption could be dealt with by adding a very small process noise to the velocity component.}
\end{itemize}
\onslide<7->{\center --------------------------------------------------------------------}
\end{frame}

\begin{frame}
  \onslide<1->{\frametitle{Non-accelerated Motion (constant velocity)}
	\framesubtitle{Covariance Extrapolation}}
\begin{itemize}
  \onslide<2->{\item $\altcovextrapeq$ becomes:  }
\begin{align*} 
\onslide<3->{\covmat{k}{-} 
&=		\begin{bmatrix}
			1 & \Delta t\\
			0 & 1
		\end{bmatrix} 
		\begin{bmatrix}
			p_{11} & p_{12}\\
			p_{21} & p_{22}
		\end{bmatrix}
		\begin{bmatrix}
			1 & 0\\
			\Delta t & 1
		\end{bmatrix}
	}\\
\onslide<4->{
&= 		\begin{bmatrix}
			p_{11}+p_{21}\Delta t & p_{12}+p_{22}\Delta t\\
			p_{21} & p_{22}
		\end{bmatrix}
		\begin{bmatrix}
		1 & 0\\
		\Delta t & 1
		\end{bmatrix}
	}\\
\onslide<5->{
&=		\begin{bmatrix}
			p_{11}+\left(p_{12}+p_{21}\right)\Delta t + p_{22}\Delta t^2 & p_{12}+p_{22}\Delta t\\
			p_{12}+p_{22}\Delta t & p_{22}
		\end{bmatrix}
	}
\end{align*} 
\onslide<6->{\item Note that we've decluttered by suppressing the +/- superscripts and $k$'s on the right-hand sides   }
\onslide<7->{\item The Kalman gain computation is left as an exercise! }
\end{itemize}
\onslide<8->{\center --------------------------------------------------------------------}
\end{frame}


\begin{frame}
  \onslide<1->{\frametitle{Non-accelerated Motion (constant velocity)} 
  \framesubtitle{Filter Performance (without Process Noise)}}
  \onslide<2->{
\begin{figure}
  \includegraphics[width=0.85\textwidth]{\dayonesegtwo/const_vel.png}
\end{figure}
}
\onslide<3->{\center --------------------------------------------------------------------}
\end{frame}


\begin{frame}
  \onslide<1->{\frametitle{Non-accelerated Motion (constant velocity)}
	\framesubtitle{Filter Sawtooth}}
Note uptick in uncertainty even without process noise, due to $\Phi\neq I$ in covariance extrapolation; causes increase in uncertainty:
  \onslide<2->{
\begin{equation*}
  \altcovextrapeq
\end{equation*}
}
  \onslide<3->{
\begin{figure}
  \includegraphics[width=0.90\textwidth,height=5cm]{\dayonesegtwo/const_vel_sawtooth.png}
%  \includegraphics[width=0.85\textwidth]{\dayonesegtwo/const_vel_sawtooth.png}
\end{figure}
}
\onslide<4->{\center --------------------------------------------------------------------}
\end{frame}

\begin{frame}
  \onslide<1->{\frametitle{Non-accelerated Motion (constant velocity)}
	\framesubtitle{Filter Performance (with Process Noise)}}
  \onslide<2->{
\begin{figure}
  \includegraphics[width=0.85\textwidth]{\dayonesegtwo/const_vel_w_proc_noise.png}
\end{figure}
}
\onslide<3->{\center --------------------------------------------------------------------}
\end{frame}


\begin{frame}
  \onslide<1->{\frametitle{Non-accelerated Motion (constant velocity)}
  \framesubtitle{Filter Sawtooth with Process Noise}}
  \onslide<2->{
\begin{figure}
  \includegraphics[width=0.85\textwidth]{\dayonesegtwo/const_vel_sawtooth_w_proc_noise.png}
\end{figure}
}
\onslide<3->{\center --------------------------------------------------------------------}
\end{frame}

\begin{frame}
  \onslide<1->{\frametitle{Non-accelerated Motion (constant velocity)}
  \framesubtitle{Filter Performance (100 samples; less Process Noise)}}
  \onslide<2->{
\begin{figure}
  \includegraphics[width=0.85\textwidth]{\dayonesegtwo/const_vel_w_less_proc_noise_100_samples.png}
\end{figure}
}
\onslide<3->{\center --------------------------------------------------------------------}
\end{frame}


\begin{frame}
  \onslide<1->{\frametitle{Non-accelerated Motion (constant velocity)}
  \framesubtitle{Filter Performance (100 samples; less Process Noise)}}
  \onslide<2->{
\begin{figure}
  \includegraphics[width=0.85\textwidth]{\dayonesegtwo/const_vel_sawtooth_w_proc_noise_100_samples.png}
\end{figure}
}
\onslide<3->{\center --------------------------------------------------------------------}
\end{frame}

\begin{frame}
  \onslide<1->{\frametitle{Non-accelerated Motion (constant velocity)}  
  \framesubtitle{Filter Performance (30 samples; large Process Noise)}}
  \onslide<2->{
\begin{figure}
  \includegraphics[width=0.85\textwidth]{\dayonesegtwo/const_vel_w_large_proc_noise_30_samples.png}
\end{figure}
}
\onslide<3->{\center --------------------------------------------------------------------}
\end{frame}

\begin{frame}
  \onslide<1->{\frametitle{Non-accelerated Motion (constant velocity)}
  \framesubtitle{Filter Sawtooth with Process Noise (30 samples; large Process Noise)}}
  \onslide<2->{
\begin{figure}
  \includegraphics[width=0.85\textwidth]{\dayonesegtwo/const_vel_sawtooth_w_large_proc_noise_30_samples.png}
\end{figure}
}
\onslide<3->{\center --------------------------------------------------------------------}
\end{frame}

\begin{frame}
  \onslide<1->{\frametitle{Non-accelerated Motion (constant velocity)}
\framesubtitle{Filter Performance \\(30 samples; large Process Noise; large Measurement Noise)}}
  \onslide<2->{\begin{figure}
  \includegraphics[width=0.85\textwidth]{\dayonesegtwo/const_vel_w_large_proc_noise_large_meas_noise_30_samples.png}
\end{figure}
}
\onslide<3->{\center --------------------------------------------------------------------}
\end{frame}

\begin{frame}
  \onslide<1->{\frametitle{Non-accelerated Motion (constant velocity)}
  \framesubtitle{Filter Sawtooth with Process Noise \\ 
  (30 samples; large Process Noise; large Measurement Noise)}}

  \onslide<2->{
\begin{figure}
  \includegraphics[width=0.85\textwidth]{\dayonesegtwo/const_vel_sawtooth_w_large_proc_noise_large_meas_noise_30_samples.png}
\end{figure}
}
\onslide<3->{\center --------------------------------------------------------------------}
\end{frame}



\begin{frame}
  \onslide<1->{
	\frametitle{Summary} 
	\framesubtitle{Vector Check}
  }

\begin{itemize}
  \onslide<2->{
\item Where are we?  
}

\begin{itemize}

\onslide<3->{\item We've examined the Constant Velocity model with 2 states and a single measurement }
\onslide<4->{\item We've shown how we can change the assumptions regarding the process in 
	the truth vs. that in the Kalman model}

\end{itemize}

\onslide<5->{
\item What's next?  
}
\begin{itemize}
  \onslide<6->{ \item In the next video, we will present the third model, the damped harmonic oscillator }
\end{itemize}

\end{itemize}
\onslide<7->{\center --------------------------------------------------------------------}
\end{frame}

