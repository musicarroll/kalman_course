% \dayonesegtwo/equation_main.tex
\begin{frame}
  \onslide<1->{\frametitle{Part I}
  \framesubtitle{The Five Basic Kalman Equations\\
  Topics}
  }
 \begin{itemize}
  \onslide<2->{\item Understanding the Equations: Heuristic Introduction}
  \onslide<3->{\item \textbf{Equation Drilldown:  Taking the Equations Apart}}
%  \onslide<4->{\item Differential Equations, Difference Equations and Dynamic Systems}
  \onslide<4->{\item State Space Concepts}
\end{itemize}
\onslide<5->{\center --------------------------------------------------------------------}
\end{frame}


\begin{frame}

\onslide<1->{
  \frametitle{Equation Drilldown:  Taking the Equations Apart}
     \framesubtitle{Topics}
}

\begin{itemize}
  \onslide<2->{\item Mathematical Formulation of the Problem}
  \onslide<3->{\item Drilldown on State Dynamics and Covariance Extrapolation Equations}
  \onslide<4->{\item The Five Kalman Filter Equations}
  \onslide<5->{\item \textbf{Examples:  Resistor Revisited}}
  \onslide<6->{\item Exercises}
\end{itemize}

\onslide<7->{\center --------------------------------------------------------------------}
\end{frame}



\begin{frame}
  \onslide<1->{\frametitle{Scalar Example: Resistor Revisited}}
\begin{itemize}
  \onslide<2->{  \item Number of states: 1\\ $x(k)$ represents resistance in Ohms at time step $k$}
  \onslide<3->{  \item System Dynamics: $x(k)=x(k-1)$ }
  \onslide<4->{  \item Measurement Model: $z(k)=x(k)+v(k)$, where $v(k)$ is zero mean, 
Gaussian white noise with variance $\sigma^2$}
  \onslide<5->{  \item Thus, $\Phi  = 1$, $H_k=1$, $Q_k=0$, $R_k=\sigma^2$}
\end{itemize} 
\onslide<6->{\center --------------------------------------------------------------------}
\end{frame}


\begin{frame}
  \onslide<1->{\frametitle{Scalar Example: Resistor Revisited}}
\begin{itemize}
   \onslide<2->{ \item State extrapolation and covariance extrapolation are simple:
\[\stateest{k}{-}=\stateest{k-1}{+}\]
\[\covmat{k}{-}=\covmat{k-1}{+}\]
where $P$ is a $1\times 1$ matrix or scalar.}
  \onslide<3->{  \item The Kalman gain equation $\kalgaineq$ becomes:
\[K_k=\covmat{k}{-}\left(\covmat{k}{-}+\sigma^2\right)^{-1}\]}
\end{itemize} 
\onslide<4->{\center --------------------------------------------------------------------}
\end{frame}

\begin{frame}
  \onslide<1->{\frametitle{Scalar Example: Resistor Revisited}}
\begin{itemize}
  \onslide<2->{  \item State update equation $\stateupdateeq$ becomes
\[\stateest{k}{+}=\stateest{k}{-}+\dfrac{\covmat{k}{-}}{\covmat{k}{-}+\sigma^2}\left(z(k)-\stateest{k}{-}\right)\]}
  \onslide<3->{  \item Covariance update equation $\covupdateeq$ becomes
\[\covmat{k}{+}=\left(1-\dfrac{\covmat{k}{-}}{\covmat{k}{-}+\sigma^2}\right)\covmat{k}{-}\]}
\end{itemize} 
\onslide<4->{\center --------------------------------------------------------------------}
\end{frame}

\begin{frame}
  \onslide<1->{\frametitle{Scalar Resistor Example Using Kalman Filter }}
   \onslide<2->{
\begin{figure}
\begin{center}
  \includegraphics[width=6cm]{\dayonesegtwo/kf_resistor_plot.png}
\end{center}
\end{figure}
}
\onslide<3->{\center --------------------------------------------------------------------}
\end{frame}

\begin{frame}
  \onslide<1->{\frametitle{Static, Scalar Example: Resistor Revisited: \\ Comparing KF and Running Average}}
   \onslide<2->{
\begin{figure}

    \hfill
    \begin{subfigure}[b]{0.45\textwidth}
      \includegraphics[width=0.95\textwidth]{\graphics/runavgplot.png}
      \caption{{\tiny Simple Running Average Estimator}}
    \end{subfigure}
    \hfill
    \begin{subfigure}[b]{0.45\textwidth}
      \includegraphics[width=0.95\textwidth]{\dayonesegtwo/kf_resistor_plot.png}
      \caption{{\tiny Kalman Estimator}}
    \end{subfigure}
\caption{{\small Comparing Running Average and Kalman Estimators}}

\end{figure}

}
\onslide<3->{\center --------------------------------------------------------------------}
\end{frame}


\begin{frame}
  \onslide<1->{\frametitle{Kalman Sawtooth Plots}}
 With constant dynamics ($\Phi=I$) and no process noise ($Q=0$) $\rightarrow$ no upticks:
   \onslide<2->{
\begin{figure}
  \includegraphics[width=0.75\textwidth]{\dayonesegtwo/kf_resistor_variance_no_process_noise.png}
\end{figure}
}
\onslide<3->{\center --------------------------------------------------------------------}
\end{frame}

\begin{comment}
\begin{frame}[fragile]
  \frametitle{Python Code Example}
  
\begin{lstlisting}[language=Python]
# Python code goes here
print("Hello, World!")
\end{lstlisting}
  
\end{frame}

\end{comment}



\begin{frame}
  \onslide<1->{
	\frametitle{Summary} 
	\framesubtitle{Vector Check}
  }

\begin{itemize}
  \onslide<2->{
\item Where are we?  
}

\begin{itemize}

\onslide<3->{\item We reformulated the resistor model as a very simple Kalman filter }
\onslide<4->{\item We've looked at the plots produced by kf\_resistor\_demo.py and seen that 
they are almost identical to the running average plots}

\end{itemize}

\onslide<6->{
\item What's next?  
}
\begin{itemize}
  \onslide<7->{ \item In the next video, we will present some exercises for you to solve. }
\end{itemize}

\end{itemize}
\onslide<8->{\center --------------------------------------------------------------------}
\end{frame}

