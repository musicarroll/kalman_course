% day1_seg2_equations/exercises.tex

\begin{frame}

  \onslide<1->{\frametitle{Exercises
}}
\begin{enumerate}
  \onslide<2->{\item Work through the resistor example by hand using 
$\hat{x}_0=100\Omega$, $\sigma=0.1$, and $P_0=0$.  Start the recursive loop 
with the gain equation first.  Run through the loop three times.}

  \onslide<3->{
\item Draw a rough plot of the error covariance, plotting both the extrapolated and 
updated covariance for any give $k$ at the same time.  (This will yield the classic 
Kalman sawtooth pattern.)
  Similarly, draw a rough plot of the state estimates.}

\setcounter{exercisecounter}{\value{enumi}}
\end{enumerate}
\end{frame}


\begin{frame}


  \onslide<1->{\frametitle{Exercises
}}
\begin{enumerate}
\setcounter{enumi}{\value{exercisecounter}}
  \onslide<2->{\item Calculate the Kalman gain of the non-accelerated motion example.  Hint:  It looks a lot like the 
gain for the damped harmonic oscillator.}
  \onslide<3->{\item Write out the state update equation for the damped harmonic oscillator, 
using the Kalman gain derived in the example.

}
  \onslide<4->{\item Write out the covariance update 
equation for the damped harmonic oscillator, using the Kalman gain dervied in the 
example.
 }
\end{enumerate}
\end{frame}
