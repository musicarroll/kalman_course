\begin{frame}
  \onslide<1->{\frametitle{Damped Harmonic Oscillator}
  \framesubtitle{Spring, Mass, Damper System}}
  \onslide<2->{
\begin{figure}
  \includegraphics[width=0.95\textwidth]{\dayonesegtwo/damped_harmonic_oscillator.png}
\end{figure}
}
\onslide<3->{\center --------------------------------------------------------------------}
\end{frame}


\begin{frame}
  \onslide<1->{\frametitle{Damped Harmonic Oscillator} 
  \framesubtitle{System Dynamics}}
\begin{itemize}
  \onslide<2->{\item System model:  Number of states = 2 (position and velocity) }
  \onslide<3->{\[\begin{bmatrix}
x_{1}(t)\\
x_{2}(t)
\end{bmatrix}=
\exp
\left(
\begin{bmatrix}
0 & 1\\
-b/m & -k/m
\end{bmatrix}
 \Delta t
\right)
\begin{bmatrix}
x_{1}(t-1)\\
x_{2}(t-1)
\end{bmatrix}
+
\begin{bmatrix}
w_{1}(t)\\
w_{2}(t)
\end{bmatrix}
\]}
    \onslide<4->{\item Thus, $\stm{t}=\exp
\left(
\begin{bmatrix}
0 & 1\\
-b/m & -k/m
\end{bmatrix}
 \Delta t
\right)$.   }
  \onslide<5->{\item Note that $\exp$ is a matrix exponential. We'll explain later how to derive state transition 
matrices from the underlying differential equations.}
  \onslide<6->{\item Here $\Delta t=T_t - T_{t-1}$, $m$ is the mass, $k$ is the 
spring constant, and $b$ is the damping constant}
\end{itemize} 
\onslide<7->{\center --------------------------------------------------------------------}
\end{frame}

\begin{frame}
  \onslide<1->{\frametitle{Damped Harmonic Oscillator}
  \framesubtitle{Process Noise}}
\begin{itemize}
  \onslide<2->{\item Assume that the process noise $w$ has the following statistics:
\[Q_t=\begin{bmatrix}
q_{11} & 0\\
0 & q_{22}
\end{bmatrix}
\]}
  \onslide<3->{  \item This means that there is some process noise in both position and velocity.}
  \onslide<4->{  \item It also means that the noise sequences in position and velocity are statistically 
uncorrelated (i.e., no non-zero off-diagonal elements in $Q_t$.)}
  \onslide<5->{  \item One may think of process noise as due to unmodelled dynamic effects}
\end{itemize} 
\onslide<6->{\center --------------------------------------------------------------------}
\end{frame}

\begin{frame}
  \onslide<1->{\frametitle{Dampled Harmonic Oscillator}
   \framesubtitle{Measurement Model}}
\begin{itemize}
  \onslide<2->{  \item Measurement Model:
\[z(t) = \begin{bmatrix}1 & 0\end{bmatrix}\begin{bmatrix}
x_{1}(t)\\
x_{2}(t)
\end{bmatrix}+
v(t)
\]
where the measurement noise $v(t)$ has variance $\sigma^2$}
  \onslide<3->{  \item Thus $H_t=\begin{bmatrix}1 & 0\end{bmatrix}$}
  \onslide<4->{  \item Note that we only have a position measurement (similar to constant velocity example).}
\end{itemize} 
\onslide<5->{\center --------------------------------------------------------------------}
\end{frame}

\begin{frame}
  \onslide<1->{\frametitle{Damped Harmonic Oscillator}
  \framesubtitle{State Transistion Matrix $\Phi$}}
\begin{itemize}
  \onslide<2->{  \item $\stm{t}$ can be approximated through series expansion of 
\[\exp\begin{bmatrix}
0 & \Delta t\\
-b/m\Delta t & -k/m\Delta t
\end{bmatrix}
\]}
  \onslide<3->{  \item Or we can use Matlab's (or numpy's) expm function for specific values 
  of $k$,$b$,$m$, and $\Delta t$.}
\end{itemize} 
\onslide<4->{\center --------------------------------------------------------------------}
\end{frame}

\begin{frame}
  \onslide<1->{\frametitle{Damped Harmonic Oscillator}
  \framesubtitle{Kalman Gain}}
\begin{itemize}
  \onslide<2->{  \item Because $\Phi$ and $Q$ are matrices, the state and covariance extrapolation equations do not simplify from their general matrix form.}
  \onslide<3->{  \item However, the gain equation $\genkalgaineq$ looks like this:
\begin{align*}
K &=
\begin{bmatrix}
p_{11} & p_{12}\\
p_{21} & p_{22}
\end{bmatrix}
\begin{bmatrix}1\\
0
\end{bmatrix}
\left( \begin{bmatrix}1 & 0\end{bmatrix}\begin{bmatrix}
p_{11} & p_{12}\\
p_{21} & p_{22}
\end{bmatrix}
\begin{bmatrix}
1\\ 
0
\end{bmatrix}
+\sigma^2
\right)^{-1}\\
&=
\begin{bmatrix}
\dfrac{p_{11}}{p_{11}+\sigma^2}\\
\dfrac{p_{21}}{p_{11}+\sigma^2}
\end{bmatrix}
\end{align*}
where we have once again suppresed the +/- superscripts and time step $t$.}
\end{itemize} 
\end{frame}

\begin{frame}
  \onslide<1->{\frametitle{Damped Harmonic Oscillator Estimator}
  \framesubtitle{MATLAB/Python Run}}
  \onslide<1->{
\begin{figure}
  \includegraphics[width=0.95\textwidth,height=7cm]{\dayonesegtwo/damposc.png}
\end{figure}
}
\onslide<4->{\center --------------------------------------------------------------------}
\end{frame}

\begin{frame}
  \onslide<1->{\frametitle{Damped Harmonic Oscillator Estimator }
  \framesubtitle{Sawtooth Plots}}
  \onslide<1->{
\begin{figure}
  \includegraphics[width=0.95\textwidth,height=7cm]{\dayonesegtwo/damposc_variance.png}
\end{figure}
}
\onslide<2->{\center --------------------------------------------------------------------}
\end{frame}

\begin{frame}
  \onslide<1->{\frametitle{Damped Harmonic Oscillator Estimator}
  \framesubtitle{MATLAB/Python Run: Conservative Process Noise Estimate}}
  \onslide<1->{
\begin{figure}
  \includegraphics[width=0.95\textwidth,height=7cm]{\dayonesegtwo/damposc_conservative.png}
\end{figure}
}
\onslide<4->{\center --------------------------------------------------------------------}
\end{frame}

\begin{frame}
  \onslide<1->{\frametitle{Damped Harmonic Oscillator Estimator Sawtooth}
  \framesubtitle{MATLAB/Python Run: Conservative Process Noise Estimate}}
  \onslide<1->{
\begin{figure}
  \includegraphics[width=0.95\textwidth,height=7cm]{\dayonesegtwo/damposc_variance_conservative.png}
\end{figure}
}
\onslide<2->{\center --------------------------------------------------------------------}
\end{frame}

