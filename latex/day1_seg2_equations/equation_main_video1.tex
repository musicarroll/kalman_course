% \dayonesegtwo/equation_main.tex
\begin{frame}
  \onslide<1->{\frametitle{Part I}
  \framesubtitle{The Basic Kalman Equations\\
  Topics}
  }
 \begin{itemize}
  \onslide<2->{\item Understanding the Equations: Heuristic Introduction}
  \onslide<3->{\item \textbf{Equation Drilldown:  Taking the Equations Apart}}
%  \onslide<4->{\item Differential Equations, Difference Equations and Dynamic Systems}
  \onslide<4->{\item State Space Concepts}
\end{itemize}
\onslide<5->{\center --------------------------------------------------------------------}
\end{frame}


\begin{frame}

\onslide<1->{
  \frametitle{Equation Drilldown:  Taking the Equations Apart}
     \framesubtitle{Topics}
}

\begin{itemize}
  \onslide<2->{\item \textbf{Mathematical Formulation of the Problem}}
  \onslide<3->{\item Drilldown on State Dynamics and Covariance Extrapolation
Equations}
  \onslide<4->{\item The Five Kalman Filter Equations}
  \onslide<5->{\item Examples}
  \onslide<6->{\item Exercises}
\end{itemize}

\onslide<7->{\center --------------------------------------------------------------------}
\end{frame}

% day1_seg1_heuristic/summary.tex

\begin{frame}
  \onslide<1->{\frametitle{Mathematical Formulation of the Problem} 
  \framesubtitle{Kalman Filter Problem Summary}
  }
 \begin{itemize}
  \onslide<2->{   \item Given two models:}
    \begin{itemize}
  	\onslide<3->{\item State Dynamics Model\\ (sometimes called the System or Process Model)}
  	\onslide<4->{\item Measurement Model\\ (sometimes called the Observation Model)}
    \end{itemize}
  \onslide<5->{ \item Find a way to \textbf{optimally} combine our knowledge of the \textbf{predicted state estimate} 
   with our knowledge of the \textbf{actual measurement} to produce an \textbf{updated state estimate}}
  \begin{itemize}
    \onslide<6->{ \item I.e., find the optimal gain with which to weight the measurement residual (difference between actual and predicted measurement) 
	}
    \onslide<7->{ \item Use weighted residual to  increment or update the state estimate} 
	\onslide<8->{ \item Use Kalman gain also to update the estimation uncertainty  
	(estimation error covariance aka "covariance")}
  \end{itemize}
 \end{itemize}
\onslide<9->{\center --------------------------------------------------------------------}

\end{frame}

\begin{frame}
  \onslide<1->{
  \frametitle{Mathematical Formulation of the Problem}
    \framesubtitle{Kalman Filter Problem Summary\\General, Non-Linear, Time-Invariant,
Discrete-Time Formulation }
}
Given:
 \begin{itemize}
   \onslide<2->{ \item State Dynamics Model:  $x(k)=f[x(k-1)] + w(k)$}
   \begin{itemize}
      \onslide<3->{ \item $k$ indicates the time step;  the actual time at that time step is $t_k$, and the time increment is $\Delta t = t_k - t_{k-1}$}
      \onslide<4->{ \item $x(k)$=state vector,$\;\;f$=true dynamics function,$\;\;w(k)$=process noise}
   \end{itemize}
   \onslide<5->{ \item Measurement Model: $z(k)=h[x(k)]+v(k)$}
   \begin{itemize}
      \onslide<6->{ \item $z(k)$=measurement,$\;\;h$=measurement function, $\;\;v(k)$=measurement noise}
   \end{itemize}
  \end{itemize}
  \onslide<7->{ Find:}
 \begin{itemize}
  \onslide<8->{ \item Optimal gain $K(k)$ that allows us to update our state estimate in the following way:\\
     $\hat{x}(k) = f[\hat{x}(k-1)] + K(k)\;[\;z(k)-h(f[\hat{x}(k-1)])\;]$}
  \onslide<9->{ \item (For now, we'll ignore the update of the covariance)}
\end{itemize}
\onslide<10->{\center --------------------------------------------------------------------}
\end{frame}

\begin{frame}
  \onslide<1->{\frametitle{Mathematical Formulation of the Problem} 
  \framesubtitle{Kalman Filter Problem Summary\\
  Limitations}
  }
 \begin{itemize}
  \onslide<2->{ \item Original Kalman filter solution applies only to ...} 
  \begin{itemize}
\onslide<3-> {\item linear systems with linear measurements driven by ...} 	 
\onslide<4-> {\item zero-mean, Gaussian white noise processes (for both process and measurement noise)}
\end{itemize}
\onslide<5->{\item Assume from now on that ...}
  \begin{itemize}
  \onslide<6-> {
    \item  Our vector-valued function $f$ is a linear, vector-valued function, i.e., a matrix $\Phi$
}
  \onslide<7->{
  \item $h$ is also a linear, vector-valued function, i.e. a matrix $H$
}
  \end{itemize}

  \onslide<8->{ \item Thus, the problem is to find a gain matrix $K(k)$ such that} 
  \onslide<9->{\center $\hat{x}(k) = \Phi\hat{x}(k-1) + K(k)\;[\;z(k)-H\Phi\hat{x}(k-1)\;]$  }

\end{itemize}
\onslide<10->{\center --------------------------------------------------------------------}

\end{frame}

\begin{frame}
  \onslide<1->{\frametitle{Mathematical Formulation of the Problem}
  \framesubtitle{Kalman Filter Problem Summary\\ Linear Discrete Formulation\\
 }
}
Given:
\begin{itemize}
  \onslide<2->{\item State Dynamics Model:
  $x(k)=\Phi x(k-1)+w(k)$}
  \onslide<3->{
    \begin{itemize}
		\item $x(k)$=state vector,$\;\;\Phi $=state transition matrix, $\;\;w(k)$=process noise vector
	\end{itemize} }
  \onslide<4->{\item Measurement Model: $z(k)=H x(k)+v(k)$} 
	\begin{itemize}
		 \onslide<5->{\item $z(k)$=measurement vector,$\;\;H$=meas. matrix, $\;\;v(k)$=meas. noise vector}
	\end{itemize}
  \end{itemize}
  \onslide<6->{Find:}
\begin{itemize}
	  \onslide<7->{\item   State estimate in the form\\
 	$\hat{x}(k) = \Phi \hat{x}(k-1) + K(k)[z(k)-H \Phi \hat{x}(k-1)]$ }
\end{itemize}
\onslide<8->{\center --------------------------------------------------------------------}

\end{frame}

\begin{frame}
  \onslide<1->{\frametitle{Mathematical Formulation of the Problem}
\framesubtitle{Kalman Filter Problem Summary \\
Slight Notational Change}
}
\begin{itemize}
  \onslide<2->{  \item The term $\Phi \hat{x}(k-1)$ is the extrapolated state estimate prior to making the measurement update or correction}
  \onslide<3->{  \item To distinguish prior (extrapolated) estimate from updated estimate, we use the following notation:}
   \onslide<4->{  	\[\hat{x}^{-}(k)=\Phi \hat{x}^{+}(k-1)=\text{\textbf{predicted estimate} prior to update}\]}
 \onslide<5->{  \[\hat{x}^{+}(k)=\text{\textbf{updated state estimate}}\] }
  \onslide<6-> {\item Applying $H$ to $\hat{x}^{-}(k)$ yields the \textbf{predicted measurement}}
\end{itemize}
\onslide<7->{\center --------------------------------------------------------------------}

\end{frame}

\begin{frame}
  \onslide<1->{
    \frametitle{Mathematical Formulation of the Problem}
    \framesubtitle{Kalman Filter Problem Solution}
  }
\begin{itemize}
  \onslide<2->{\item Thus, the solution we seek now looks like this:\\}
  \onslide<3->{
    \begin{equation}\label{filtersolution}
       \hat{x}^{+}(k)=\hat{x}^{-}(k)+ K(k)[z(k)-H\hat{x}^{-}(k)]
    \end{equation}
}
\onslide<4-> {
  where $\hat{x}^{-}(k)=\Phi \hat{x}^{+}(k-1) $
}
\onslide<5-> {
  \item This now looks very much like the running average problem!
}
\end{itemize}
\onslide<6->{\center --------------------------------------------------------------------}

\end{frame}




\begin{frame}
  \onslide<1->{
	\frametitle{Summary} 
	\framesubtitle{Vector Check}
}
\begin{itemize}
  \onslide<2->{
\item Where are we?  
}
\begin{itemize}

\onslide<3->{\item We've formulated the linear discrete problem}
\onslide<4->{\item We've indicated the form of the solution: the recursive estimating algorithm with the Kalman gain}
\onslide<5->{\item But we haven't yet presented the complete solution}
\begin{itemize}
\onslide<6->{\item For the complete solution, we'll need to compute the gain $K(k)$ and use it to update both the state estimate \textbf{and} the estimation uncertainty }
\end{itemize}
\end{itemize}
  \onslide<7->{
\item What's next?  
}
\begin{itemize}
  \onslide<8->{
\item Drill down on state dynamics and determine how
estimation uncertainty changes over time and also how it is updated  
}
  \onslide<9-> {\item Use both the predicted estimation uncertainty and measurement noise uncertainty to calculate the Kalman gain}
  \onslide<10-> {\item Use Kalman gain to update estimation uncertainty}
\end{itemize}
\end{itemize}
\onslide<11->{\center --------------------------------------------------------------------}
\end{frame}

