\begin{frame}
  \onslide<1->{\frametitle{The Kalman Filter Solution}
  \framesubtitle{The Kalman Gain}
  }
\begin{itemize}
\onslide<2->{\item The heart of the solution is the \textbf{\textit{Kalman Gain}}:  $\kalgain{k}$}
\onslide<3->{
\begin{equation}\label{kalgaineq}
\kalgaineq
\end{equation}}
\onslide<4->{where $R(k)$ is the \textbf{measurement noise covariance matrix} (to be defined later) }
\begin{itemize}
	\onslide<5->{\item $R(k)$ governs the white measurement noise $v(k)$ in the measurement model:}
\end{itemize}
\onslide<6->{
\[\measmodeleq\] 
}
\onslide<7->{\item But where did the Kalman gain Eq.(\ref{kalgaineq}) come from?}
\end{itemize} 
\onslide<8->{\center --------------------------------------------------------------------}
\end{frame}

\begin{frame}
	\onslide<1->{\frametitle{The Kalman Filter Solution}
		\framesubtitle{The Kalman Gain}
	}
{\center {\scriptsize $\kalgaineq\;\;\;\;$ (\ref{kalgaineq}) }}
	\begin{itemize}
		\onslide<2->{\item We are not going to formally derive Eq.(\ref{kalgaineq}) right now.}
		\onslide<3->{\item But we will take it apart and consider why it makes sense}
		\onslide<4->{\item First consider that the expression on the right side has the form: $AB^{-1}$}
		\onslide<5->{\item This is analogous to a scalar fraction: $\dfrac{a}{b}$, i.e., a ratio}
		\onslide<6->{\item Now ignore the $H$ matrix since it only serves to map the state space-related matrix $P$ into measurement space; think of it as a scaling or projection factor}
	\end{itemize} 
	\onslide<7->{\center --------------------------------------------------------------------}
\end{frame}

\begin{frame}
	\onslide<1->{\frametitle{The Kalman Filter Solution}
		\framesubtitle{The Kalman Gain}
	}
{\center {\scriptsize $\kalgaineq\;\;\;\;$ (\ref{kalgaineq}) }}
	\begin{itemize}
		\onslide<2->{\item Note also that the matrices $P$ and $R$ are positive definite\footnote{Technically, covariance matrices only need to be positive \textbf{semi}definite.}}
		\onslide<3->{\item So, we are essentially considering the ratio of $P$ to $P+R$}
		\onslide<4->{\item But the $P$ here is the extrapolated $P$ that incorporates the previous $P$ and $Q$ (also positive definite) }
		\onslide<5->{\item So, we effectively looking at a ratio of $P+Q$ to $P+Q+R$ }
		\onslide<6->{\item If $R$ were small (a near perfect sensor), then $K$ would be near unity, and we would effectively ignore the prior state estimate and just trust the measurement; which makes sense, given a perfect sensor }
		\onslide<7->{\item At the other extreme, if $R$ were infinitely huge (a non-functional sensor), then $K$ would be near 0 and we would effectively ignore the measurement in favor of the previously predicted estimate }

	\end{itemize} 
	\onslide<8->{\center --------------------------------------------------------------------}
\end{frame}


\begin{frame}
  \onslide<1->{\frametitle{The Kalman Filter Solution }
  \framesubtitle{State Update Equation}}
\begin{itemize}
\onslide<2->{\item Using the gain Eq. (\ref{kalgaineq})\\ }
\onslide<3->{{\center $\kalgaineq$,\\}}
\onslide<4->{the state update equation (which we've already seen) is 
\begin{equation}\label{stateupdateeq}
\stateupdateeq
\end{equation}
where $z(k)$ is the measurement and $H(k)$ is the measurement matrix at time $k$.
}
\end{itemize} 
\onslide<5->{\center --------------------------------------------------------------------}
\end{frame}

\begin{frame}
  \onslide<1->{\frametitle{The Kalman Filter Solution}
  	\framesubtitle{The Covariance Update Equation}
  }
\begin{itemize}
\onslide<2->{\item Likewise, the error covariance $P(k)$ is corrected using the gain:
\begin{equation}\label{covupdateeq}
\covupdateeq
\end{equation}
} 
\onslide<3->{\item This is not a good form to use in numerical work!}
\onslide<4->{\item Better to use the Joseph form:}\\
\onslide<5->{ {\small \[ \josephcovupdateeq\] } }
\begin{itemize}
\onslide<6->{\item which better preserves symmetry of $P$\\ (covariance matrices are always symmetric)}
\end{itemize}
\end{itemize} 
\onslide<7->{\center --------------------------------------------------------------------}
\end{frame}

\begin{frame}
	\onslide<1->{\frametitle{The Kalman Filter Solution}
		\framesubtitle{Taking Apart the Covariance Update Equation}
	}
	\begin{itemize}
		\onslide<2->{\item Again, ignore $H$ in Eq.(\ref{covupdateeq}):$\;\;\covupdateeq$}
		\onslide<3->{\item Thus, $I-K$ indicates, in a sense, how far we are from having no sensors at all}
		\onslide<4->{\item If $K=0$,(i.e., no sensors at all) there would be no reduction in uncertainty }
		\onslide<5->{\item If $K=I$, we would have perfect knowledge (thanks to perfect sensors), i.e., $P=0$}
		\onslide<6->{\item Thus, if $0< K < I,$ this use of $I-K$ will tend to reduce the uncertainty $P$ }
		\onslide<7->{\item Later we will see how this form for $K$ and its use in the update equations is in fact mandated by the requirement of optimal use of $\Phi,P,Q,R$  }
		
	\end{itemize} 
	\onslide<8->{\center --------------------------------------------------------------------}
\end{frame}

\begin{comment}
\begin{frame}
  \frametitle{The Five Kalman Equations}
  \begin{tabular}{rlc}
    \onslide<1->{ {\scriptsize State Extrapolation:} & {\tiny $\stateextrapeq$  } & {\scriptsize (TBD) }} \\
    \onslide<2->{ {\scriptsize Covariance Extrapolation: }&{\tiny  $\altcovextrapeq$} & {\scriptsize (TBD) }} \\
    \onslide<3->{ {\scriptsize Kalman Gain:}& {\tiny $\kalgaineq$} & {\scriptsize \ref{kalgaineq} }} \\
    \onslide<4->{ {\scriptsize State Update:}& {\tiny $\stateupdateeq$} & {\scriptsize \ref{stateupdateeq}}} \\
    \onslide<5->{ {\scriptsize Covariance Update:}& \makecell{ {\tiny  $\covupdateeq$ or \\  $\josephcovupdateeq$ } } & {\scriptsize \ref{covupdateeq}} }
  \end{tabular}
  \onslide<6->{\center --------------------------------------------------------------------}
\end{frame}
\end{comment}

\begin{frame}
  \frametitle{The Five Kalman Equations}
\onslide<1->{\begin{equation*}
\textbf{State Extrapolation: }\stateextrapeq
\end{equation*}} 
\onslide<2->{\begin{equation*}
\textbf{Covariance Extrapolation: }\altcovextrapeq
\end{equation*}} 
\onslide<3->{\begin{equation*}
\textbf{Kalman Gain: }\kalgaineq
\end{equation*}} 
\onslide<4->{\begin{equation*}
\textbf{State Update: }\stateupdateeq
\end{equation*}} 
\onslide<5->{
	\begin{equation*}
		\textbf{Covariance Update: }\covupdateeq
	\end{equation*} 
\onslide<6->{or the Joseph form for the Covariance Update:\\}		
\onslide<7->{	{\center \small $\josephcovupdateeq$ }}	
	}

\onslide<8->{\center --------------------------------------------------------------------}
 
\end{frame}

\begin{frame}
  \onslide<1->{\frametitle{Definitions for the Five Equations }}
\begin{itemize}
  \onslide<2->{\item $\covmatdef$ is the estimation error covariance matrix (formal definition)}
  \begin{itemize}  
  \onslide<3->{\item where $E[...]$ is statistical expectation (more on that later) and }
  \onslide<4->{\item $\esterrdef$ is the estimation error  }
	\onslide<5->{\item $P$ is often just referred to as the 'covariance matrix'}
  \end{itemize}
  \onslide<6->{\item $\procnoisematdef{k}$ is the process noise covariance matrix }
  \onslide<7->{\item $\measnoisematdef{k}$ is the measurement noise covariance matrix }
\end{itemize}
\onslide<8->{\center --------------------------------------------------------------------}
\end{frame}

