\begin{frame}
  \onslide<1->{\frametitle{The Kalman Filter Solution}
  \framesubtitle{The Kalman Gain}
  }
\begin{itemize}
\onslide<2->{\item The heart of the solution is the \textbf{\textit{Kalman Gain}}:  $\kalgain{k}$}
\onslide<3->{
\begin{equation}\label{kalgaineq}
\kalgaineq
\end{equation}
where $R(k)$ is the measurement noise covariance matrix (to be defined later) governing the 
white measurement noise $v(k)$ in the measurement model:
\[\measmodeleq\] 
}
\end{itemize} 
\onslide<4->{\center --------------------------------------------------------------------}
\end{frame}

\begin{frame}
  \onslide<1->{\frametitle{The Kalman Filter Solution:  State Update Equation}}
\begin{itemize}
\onslide<2->{\item Using the gain, the state update equation (which we've already seen) is 
\begin{equation}\label{stateupdateeq}
\stateupdateeq
\end{equation}
where $z(k)$ is the measurement and $H(k)$ is the measurment matrix at time $k$.
}
\end{itemize} 
\onslide<3->{\center --------------------------------------------------------------------}
\end{frame}

\begin{frame}
  \onslide<1->{\frametitle{The Kalman Filter Solution:  The Covariance Update Equation}}
\begin{itemize}
\onslide<2->{\item Likewise, the error covariance $P(k)$ is corrected using the gain:
\begin{equation}\label{covupdateeq}
\covupdateeq
\end{equation}
} 
\onslide<3->{\item This is not a good form to use in numerical work!}
\onslide<4->{\item Better to use the Joseph form:}\\
\onslide<5->{ {\small \[ \josephcovupdateeq\] } }
\begin{itemize}
\onslide<6->{\item Better preserves symmetry of $P$}
\end{itemize}
\end{itemize} 
\onslide<7->{\center --------------------------------------------------------------------}
\end{frame}

%\begin{frame}
%  \frametitle{The Five Kalman Equations}
%  \begin{tabular}{rl}
%    \onslide<1->{ {\small State Extrapolation:} &{\small $\stateextrapeq$  } }\\
%    \onslide<2->{ {\small Covariance Extrapolation: }&{\small  $\altcovextrapeq$}} \\
%    \onslide<3->{ {\small Kalman Gain:}& {\small $\kalgaineq$}} \\
%    \onslide<4->{ {\small State Update:}& {\small $\stateupdateeq$}} \\
%    \onslide<5->{ {\small Covariance Update:}&{\small  $\covupdateeq$ }}
%  \end{tabular}
%  \onslide<6->{\center --------------------------------------------------------------------}
%\end{frame}


\begin{frame}
  \frametitle{The Five Kalman Equations}
\onslide<1->{\begin{equation}
\textbf{State Extrapolation: }\stateextrapeq
\end{equation}} 
\onslide<2->{\begin{equation}
\textbf{Covariance Extrapolation: }\altcovextrapeq
\end{equation}} 
\onslide<3->{\begin{equation}
\textbf{Kalman Gain: }\kalgaineq
\end{equation}} 
\onslide<4->{\begin{equation}
\textbf{State Update: }\stateupdateeq
\end{equation}} 
\onslide<5->{
	\begin{equation}
		\textbf{Covariance Update: }\covupdateeq
	\end{equation} 
		or {\small $\josephcovupdateeq$ }
(the Joseph form)}

\onslide<6->{\center --------------------------------------------------------------------}
 
\end{frame}

\begin{frame}
  \onslide<1->{\frametitle{Definitions for the Five Equations }}
\begin{itemize}
  \onslide<2->{\item $\covmatdef$ is the estimation error covariance matrix}
  \begin{itemize}  
	\onslide<3->{\item Often just referred to as the 'covariance matrix'}
  \end{itemize}
  \onslide<4->{\item $\esterrdef$ is the estimation error  }
  \onslide<5->{\item $\procnoisematdef{k}$ is the process noise covariance matrix }
  \onslide<6->{\item $\measnoisematdef{k}$ is the measurement noise covariance matrix }
  \onslide<7->{\item In the above, $E[...]$ is statistical expectation (more on that later) }
\end{itemize}
\onslide<8->{\center --------------------------------------------------------------------}
\end{frame}

