% day1_seg1_heuristic/summary.tex

\begin{frame}
  \onslide<1->{\frametitle{Mathematical Formulation of the Problem} 
  \framesubtitle{Kalman Filter Problem Summary}
  }
 \begin{itemize}
  \onslide<2->{   \item Given two models:}
    \begin{itemize}
  	\onslide<3->{\item State Dynamics Model\\ (sometimes called the System or Process Model)}
  	\onslide<4->{\item Measurement Model\\ (sometimes called the Observation Model)}
    \end{itemize}
  \onslide<5->{ \item Find a way to \textbf{optimally} combine our knowledge of the \textbf{predicted state} 
   with our knowledge of the \textbf{actual measurement} to produce an \textbf{updated state estimate}}
  \begin{itemize}
    \onslide<6->{ \item I.e., find the optimal linear gain $K(k)$ with which to weight the measurement residual 
		(or innovation)}
    \onslide<7->{ \item The weighted residual is then used to additively update the state estimate} 
	\onslide<8->{ \item The Kalman gain is also used to update the estimation error covariance 
	(estimation uncertainty)}
  \end{itemize}
 \end{itemize}
\onslide<9->{\center --------------------------------------------------------------------}

\end{frame}

\begin{frame}
  \onslide<1->{
  \frametitle{Mathematical Formulation of the Problem}
    \framesubtitle{Kalman Filter Problem Summary\\General, Non-Linear, Time-Invariant,
Discrete-Time Formulation\\Given: }
}
 \begin{itemize}
   \onslide<2->{ \item State Dynamics Model:  $x(k)=f[x(k-1)] + w(k)$}
   \begin{itemize}
      \onslide<3->{ \item $x$=state,$f$=true dynamics function,$w$=process noise}
   \end{itemize}
   \onslide<4->{ \item Measurement Model: $z(k)=h[x(k)]+v(k)$}
   \begin{itemize}
      \onslide<5->{ \item $z$=measurement,\;\;$h$=measurement function,\;\; $v$=measurement noise}
   \end{itemize}
  \end{itemize}
  \onslide<6->{ Find:}
 \begin{itemize}
  \onslide<7->{ \item Optimal gain $K(k)$ that allows us to update our state estimate in the following way:\\
     $\hat{x}(k) = f[\hat{x}(k-1)] + K(k)\;[\;z(k)-h(f[\hat{x}(k-1)])\;]$}
  \onslide<8->{ \item We'll ignore the form of the covariance for now!}
\end{itemize}
\onslide<9->{\center --------------------------------------------------------------------}
\end{frame}

\begin{frame}
  \onslide<1->{\frametitle{Mathematical Formulation of the Problem} \framesubtitle{Kalman Filter Problem Summary}
  }
 \begin{itemize}
  \onslide<2->{ \item Kalman filtering only deals with linear (or linearized) system dynamics
  \begin{itemize}
  \onslide<3->{
    \item Assume from now on that vector-valued function $f[\hat{x}(k-1)]$ is simply
matrix multiplication, something like $\Phi$ \\(which we assume for now does not vary with time)
}
  \onslide<4->{
  \item Also assume that $h$ is a linear, vector-valued function, i.e. a matrix $H$
}
  \end{itemize}
}
  \onslide<5->{ \item Find a gain $K(k)$ in form of multiplier (i.e., a gain matrix) such that} \\
  \onslide<6->{\center $\hat{x}(k) = \Phi\hat{x}(k-1) + K(k)\;[\;z(k)-H\Phi\hat{x}(k-1)\;]$  \\}
  \onslide<7->{ ( {\small We'll declutter this soon by giving $\Phi\hat{x}(k-1)$ a new notation })}

\end{itemize}
\onslide<8->{\center --------------------------------------------------------------------}

\end{frame}

\begin{frame}
  \onslide<1->{\frametitle{Mathematical Formulation of the Problem}
  \framesubtitle{Kalman Filter Problem Summary\\ (Linear Discrete Formulation)\\
Given: }
}
\begin{itemize}
  \onslide<2->{\item State Dynamics Model:
  $x(k)=\Phi x(k-1)+w(k)$}
  \onslide<3->{
    \begin{itemize}
		\item $x(k)$=state,$\Phi $=state transition matrix, $w(k)$=process noise
	\end{itemize} }
  \onslide<4->{\item Measurement Model: $z(k)=H x(k)+v(k)$ \\
	\begin{itemize}
		\item $z(k)$=measurement,$H$=meas. matrix (assumed constant), $v(k)$=meas. noise
	\end{itemize}}
  \end{itemize}
  \onslide<5->{Find:}
\begin{itemize}
	  \onslide<6->{\item   State estimate in the form\\
 	$\hat{x}(k) = \Phi \hat{x}(k-1) + K(k)[z(k)-H \Phi \hat{x}(k-1)]$ }
	  \onslide<7->{\item Note that $\Phi \hat{x}(k-1)$ extrapolates of the previous estimate to the current time}
\end{itemize}
\onslide<8->{\center --------------------------------------------------------------------}

\end{frame}

\begin{frame}
  \onslide<1->{\frametitle{Mathematical Formulation of the Problem}
\framesubtitle{Kalman Filter Problem Summary \\
Slight Notational Change}
}
\begin{itemize}
  \onslide<2->{  \item The term $\Phi \hat{x}(k-1)$ is the extrapolated state estimate prior to making the measurement update or correction}
  \onslide<3->{  \item To distinguish prior (extrapolated) estimate from updated estimate, we use the following notation:\
\[\hat{x}^{-}(k)=\Phi \hat{x}^{+}(k-1)=\text{predicted estimate prior to update}\]
\[\hat{x}^{+}(k)=\text{updated state estimate}\] }
\end{itemize}
\onslide<4->{\center --------------------------------------------------------------------}

\end{frame}

\begin{frame}
  \onslide<1->{
    \frametitle{Mathematical Formulation of the Problem}
    \framesubtitle{Kalman Filter Problem Summary}
  }
\begin{itemize}
  \onslide<2->{\item Thus, the solution we seek now looks like this:\\}
  \onslide<3->{
    \begin{equation}\label{filtersolution}
       \hat{x}^{+}(k)=\hat{x}^{-}(k)+ K(k)[z(k)-H\hat{x}^{-}(k)]
    \end{equation}
}
\onslide<4-> {
  where $\hat{x}^{-}(k)=\Phi \hat{x}^{+}(k-1) $
}
\onslide<5-> {
  \item This now looks a lot like the running average problem!
}
\end{itemize}
\onslide<6->{\center --------------------------------------------------------------------}

\end{frame}

