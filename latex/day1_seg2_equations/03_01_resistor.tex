\begin{frame}
  \onslide<1->{\frametitle{Scalar Example: Resistor Revisited}}
\begin{itemize}
  \onslide<2->{  \item Number of states: 1}
  \onslide<3->{  \item System Dynamics: $x(k)=x(k-1)$}
  \onslide<4->{  \item Measurement Model: $z(k)=x(k)+v(k)$, where $v(k)$ is zero mean, 
Gaussian white noise with variance $\sigma^2$}
  \onslide<5->{  \item Thus, $\Phi  = 1$, $H_k=1$, $Q_k=0$, $R_k=\sigma^2$}
\end{itemize} 
\onslide<6->{\center --------------------------------------------------------------------}
\end{frame}

\begin{frame}
  \onslide<1->{\frametitle{Scalar Example: Resistor Revisited}}
\begin{itemize}
   \onslide<2->{ \item State extrapolation and covariance extrapolation are simple:
\[\stateest{k}{-}=\stateest{k-1}{+}\]
\[\covmat{k}{-}=\covmat{k-1}{+}\]
where $P$ is a $1\times 1$ matrix or scalar.}
  \onslide<3->{  \item The Kalman gain equation $\kalgaineq$ becomes:
\[K_k=\covmat{k}{-}\left(\covmat{k}{-}+\sigma^2\right)^{-1}\]}
\end{itemize} 
\onslide<4->{\center --------------------------------------------------------------------}
\end{frame}

\begin{frame}
  \onslide<1->{\frametitle{Scalar Example: Resistor Revisited}}
\begin{itemize}
  \onslide<2->{  \item State update equation $\stateupdateeq$ becomes
\[\stateest{k}{+}=\stateest{k}{-}+\dfrac{\covmat{k}{-}}{\covmat{k}{-}+\sigma^2}\left(z(k)-\stateest{k}{-}\right)\]}
  \onslide<3->{  \item Covariance update equation $\covupdateeq$ becomes
\[\covmat{k}{+}=\left(1-\dfrac{\covmat{k}{-}}{\covmat{k}{-}+\sigma^2}\right)\covmat{k}{-}\]}
\end{itemize} 
\onslide<4->{\center --------------------------------------------------------------------}
\end{frame}

\begin{frame}
  \onslide<1->{\frametitle{Scalar Resistor Example Using Kalman Filter }}
   \onslide<2->{
\begin{figure}
\begin{center}
  \includegraphics[width=6cm]{\dayonesegtwo/kf_resistor_plot.png}
\end{center}
\end{figure}
}
\onslide<3->{\center --------------------------------------------------------------------}
\end{frame}

\begin{frame}
  \onslide<1->{\frametitle{Static, Scalar Example: Resistor Revisited: \\ Comparing KF and Running Average}}
   \onslide<2->{
\begin{figure}

    \hfill
    \begin{subfigure}[b]{0.45\textwidth}
      \includegraphics[width=\textwidth]{\graphics/runavgplot.png}
      \caption{Simple Running Average Estimator}
    \end{subfigure}
    \hfill
    \begin{subfigure}[b]{0.45\textwidth}
      \includegraphics[width=\textwidth]{\dayonesegtwo/kf_resistor_plot.png}
      \caption{Kalman Estimator}
    \end{subfigure}
\caption{Comparing Running Average and Kalman Estimators}

\end{figure}

}
\onslide<3->{\center --------------------------------------------------------------------}
\end{frame}


\begin{frame}
  \onslide<1->{\frametitle{Kalman Sawtooth Plots}}
By interleaving the pre- and post-update elements of $P$ over a run (typically only 
the diagonal elements), we can obtain Kalman sawtooth plots.
   \onslide<2->{
\begin{figure}
  \includegraphics[width=0.75\textwidth]{\dayonesegtwo/kf_resistor_variance_w_process_noise.png}
\end{figure}
}
\onslide<3->{\center --------------------------------------------------------------------}
\end{frame}

\begin{frame}
  \onslide<1->{\frametitle{Kalman Sawtooth Plots}}
Plots show growth in uncertainty (upticks) upon time extrapolation and reduction in 
uncertainty (downticks) after measurement updates.  No process noise $\rightarrow$ no upticks:
   \onslide<2->{
\begin{figure}
  \includegraphics[width=0.75\textwidth]{\dayonesegtwo/kf_resistor_variance_no_process_noise.png}
\end{figure}
}
\onslide<3->{\center --------------------------------------------------------------------}
\end{frame}


